\begin{tikzpicture}[font=\small, node distance=5mm and 16mm]

% --- Stili per i box principali ---
\tikzset{
  groupbox/.style={inner sep=0pt},
  grouplabel/.style={font=\footnotesize\bfseries, fill=white, inner sep=1pt}
}

\def\WNode{45mm}
\def\HNode{25mm}
\def\NDistanceH{10mm}
\def\NDistanceV{2mm}

% =============== BLOCCO SINISTRA (SX): 3 elementi ===============
\node[layer, minimum width=\WNode] (L1) {\textbf{Chat interface}\\Claude Desktop};
\node[layer, minimum width=\WNode, below=\NDistanceV of L1] (L2) {\textbf{IDEs and code editors}\\Claude Code, Goose};
\node[layer, minimum width=\WNode, below=\NDistanceV of L2] (L3) {\textbf{Other AI applications}\\5ire, Superinterface};
% Box che "abbraccia" i tre nodi: il suo centro è a metà verticale
\node[groupbox, fit=(L1) (L3), label={[grouplabel, label distance=\NDistanceV]south:\gls{mcp} Hosts}] (SXBOX) {};

% =============== BLOCCO CENTRALE (CENTRO): 1 elemento ===========
% Posizionato "right=..." RISPETTO A SXBOX (non L1): così condivide la stessa y del centro
\node[layer, fill=cHAL!60, minimum width=\WNode, minimum height=\HNode, right=\NDistanceH of SXBOX] (C1)
  {\textbf{MCP}\\Standardized protocol};
\node[groupbox, fit=(C1)] (CTBOX) {};

% =============== BLOCCO DESTRA (DX): 3 elementi =================
% Posiziona il blocco destro a destra del nodo centrale (che è ora centrato verticalmente)
\node[layer, minimum width=\WNode, right={\NDistanceH*2+\WNode} of L1] (R1) {\textbf{Data and file systems}\\PostgreSQL, SQLite, GDrive};
\node[layer, minimum width=\WNode, below=\NDistanceV of R1] (R2) {\textbf{Development tools}\\Git, Sentry, etc.};
\node[layer, minimum width=\WNode, below=\NDistanceV of R2] (R3) {\textbf{Productivity tools}\\Slack, Google Maps, etc.};
\node[groupbox, fit=(R1) (R3), label={[grouplabel, label distance=\NDistanceV]south:\gls{mcp} Server}] (DXBOX) {};

% --- Stile frecce ---
\tikzset{
  bidi/.style={
    draw=black!35,
    line width=0.5pt,
    rounded corners=2mm,
    {Latex[length=2mm]}-{Latex[length=2mm]}
  },
}


\def\ADistanceH{\NDistanceH*0.5}
\def\ADistanceV{6mm}

\draw[bidi] (L1.east) -|  ([shift=({-\ADistanceH, \ADistanceV})]C1.west) -- ([yshift={\ADistanceV}]C1.west);
\draw[bidi] (L2.east) -- (C1.west);
\draw[bidi] (L3.east) -|  ([shift=({-\ADistanceH, -\ADistanceV})]C1.west) -- ([yshift={-\ADistanceV}]C1.west);


\draw[bidi] ([yshift={\ADistanceV}]C1.east) --  ([shift=({\ADistanceH, \ADistanceV})]C1.east) |- (R1.west);
\draw[bidi] (C1.east) -- (R2.west);
\draw[bidi] ([yshift={-\ADistanceV}]C1.east) --  ([shift=({\ADistanceH, -\ADistanceV})]C1.east) |- (R3.west);




\end{tikzpicture}
