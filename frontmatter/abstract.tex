%!TEX root = ../dissertation.tex
Android applications often integrate native code, such as C/C++, for improved performance or specific features. When an app uses the Java Native Interface (JNI) to call this code, however, the app can be exposed to vulnerabilities that arise in the native layer. Although native libraries can improve performance, C/C++ lacks the memory-safety guarantees of Java and Kotlin. This makes issues such as buffer overflows, use-after-free errors, and other memory corruption problems more likely. A weakness in the native component can compromise the entire application, even if the Java side is written securely.

Tools such as POIROT can automatically generate harnesses and fuzz native libraries to search for crashes. However, these tools still rely on manual crash triage to determine if a crash indicates a vulnerability. This thesis proposes an LLM-based triaging pipeline that uses POIROT crash reports to automatically estimate the security risk of applications.

This approach uses the Model Context Protocol (MCP) to provide the language model with controlled access to reverse-engineering tools: Jadx is used for Android bytecode/manifest context, and Ghidra is used for native disassembly/decompilation. Using crashes produced by POIROT as input, the pipeline analyzes the origin of each crash, returns a classification stating whether the issue is likely a vulnerability and the level of confidence in that judgment, and explains the reasons in plain terms. It also indicates a severity level and anchors its reasoning with concrete evidence, such as specific stack frames, function names, or short decompiled snippets. The pipeline also suggests practical next steps or mitigations and explicitly notes any assumptions and limitations, highlighting what information was missing or uncertain.

The goal is to reduce analyst effort and improve consistency by automating crash triage with verifiable context from MCP tools. This thesis presents the design and implementation of this approach, as well as the obtained results.