\chapter{Conclusions}
\label{chp:conclusions}

This thesis presented an automated triage pipeline that combines Large Language Models with the Model Context Protocol (MCP) to analyse native crashes in Android applications.  
The goal was to support vulnerability assessment in settings where manual crash inspection is costly, and where modern applications integrate substantial amounts of native code exposed through the Java Native Interface.

The proposed system integrates reverse-engineering tools---Jadx for bytecode analysis and Ghidra for native disassembly---into the LLM’s reasoning loop, enabling context-aware classification of POIROT-generated crashes.  
The evaluation demonstrates that the pipeline reliably identifies real vulnerabilities, achieving consistently low false-negative rates and providing structured explanations, confidence scores, and evidence traces.  
The tpCamera case study further shows that the system can approximate expert-level analysis: the LLM classification aligns with the ground-truth vulnerability reported in PIROT (CVE-2023-30273), correctly identifying the unsafe lifecycle of the encoder context and the conditions leading to a use-after-free.

At the same time, the results highlight several limitations.  
Precision remains modest, primarily due to incomplete crash traces and the conservative behaviour of general-purpose LLMs, which tend to over-approximate risk when information is missing.  
Additional restrictions arise from partial decompilation, the heterogeneity of Android JNI patterns, and the uncertainty inherent to LLM-based reasoning.  
These findings indicate that such a system should not replace traditional analyses, but rather act as an initial filter that accelerates and prioritises manual investigation.

Despite these limitations, the work demonstrates the viability of using LLMs to enhance automated vulnerability triage.  
The approach scales to applications with numerous native libraries, reduces the manual burden on analysts, and provides reproducible reasoning grounded in explicit code evidence.  
This opens opportunities for integrating LLM-assisted triage into broader fuzzing pipelines or continuous security testing frameworks.

Future work should focus on improving precision through richer static and dynamic context, extending Java–to–native call-graph reconstruction, and exploring ensemble or multi-agent architectures to reduce uncertainty.  
Integrating automated exploitability assessment and symbolic reasoning layers could also further strengthen the system’s reliability.  

In conclusion, this thesis shows that LLM-based crash triage, when combined with reverse-engineering tools through MCP, offers a promising direction for scalable vulnerability analysis.  
It enhances the efficiency and consistency of the triage process and provides a practical foundation for more advanced, context-aware security automation.
