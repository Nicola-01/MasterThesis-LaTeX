\chapter{Implementation}
\label{chp:impl}

The triage system, implemented in Python3, orchestrates an \gls{llm}-based tool over crashes discovered by the fuzzing pipeline(see chapter~\ref{chp:preliminaries} (\emph{Preliminaries})). Conceptually, the program:\\
(i) ingests POIROT output (i.e. stack traces).\\
(ii) enriches them with reverse-engineering context via the \gls{mcp} (Jadx for bytecode/manifest, Ghidra for native disassembly/decompilation).\\
(iii) produces a structured judgement comprising a vulnerability likelihood, a succinct rationale anchored to concrete evidence (frames, symbols, code hunks), and a severity estimate. 

The implementation emphasises strict data modelling, explicit tool adaptors, and prompt templates specialised for each tool, so that the resulting traces are reproducible and auditable.

\textcolor{red}{Da rivedere quando il capitolo è finito}


\section{Libraries and Dependencies}
\textcolor{red}{Da vedere}
The codebase relies on a small set of focused Python libraries and external tools:
\begin{itemize}
  \item \texttt{pydantic} (v2.12.3) for strict, typed data models (e.g., crash summaries and tool responses) and runtime validation. This allows us to require that the \gls{llm} return an instance conforming to a predefined model, so the output can be parsed deterministically (without ad-hoc prompt parsing) and serialised/deserialised as JSON with minimal glue code.
  \item \texttt{pydantic\_ai} for agent composition and the \emph{\gls{mcp}} client; specifically, the \texttt{MCPServerStdio} transport is used to connect to tool servers over stdio.
  %\item \textbf{LLM providers.} The agent is provider-agnostic and supports OpenAI/Google backends; a local wrapper (\texttt{ollamaLocal.py}) is included for offline experiments.
  %\item \textbf{External tools (via MCP).} Jadx and Ghidra are accessed through dedicated \gls{mcp} adaptors; prompts and post-processing live under \texttt{MCPs/} (see below).
\end{itemize}
All configuration (model choice, tool endpoints, timeouts) is read at start-up; failures in any tool adaptor fail fast with descriptive error messages and are surfaced in the final report.

\section{Program Structure}
Figure~\ref{fig:program-structure} summarises the modules and the directory layout used during development.

\paragraph{Entry point and orchestration.}
\texttt{main.py} parses inputs (paths to POIROT outputs, model/tool configuration), instantiates the analysis \emph{agent} via \texttt{MCPs/get\_agent.py}, and coordinates the end-to-end run: load crash artefacts $\rightarrow$ enrich with tool context $\rightarrow$ run triage $\rightarrow$ emit a structured report.

\paragraph{Data model and utilities.}
\texttt{CrashSummary.py} defines the canonical in-memory representation for a crash (stack frames, symbol/offset, JNI entry point, reproducer, harness sequence, metadata). \texttt{utils.py} centralises I/O (filesystem paths, JSON/NDJSON log handling) and small helpers (e.g., normalising symbol names, safer subprocess calls). \texttt{jadx\_helper\_functions.py} offers convenience wrappers for common Jadx queries (e.g., resolve a method signature, fetch call sites, extract a minimal decompiled hunk).

\paragraph{Agent and MCP adaptors.}
\texttt{MCPs/get\_agent.py} composes the triage agent with the required tool capabilities and the selected \gls{llm} provider. \texttt{MCPs/jadxMCP.py} and \texttt{MCPs/ghidraMCP.py} implement thin clients over \texttt{MCPServerStdio}, handling request/response schemas, error propagation, and small post-processing (e.g., trimming code hunks and attaching file/offset provenance). \texttt{MCPs/ollamaLocal.py} is an optional backend to run models locally. \texttt{MCPs/vulnAssessment.py} aggregates model/tool outputs into the final, typed judgement (likelihood, confidence, CWE/notes, severity).

\paragraph{Prompt specialisation.}
Under \texttt{MCPs/prompts/}, \texttt{ghidra\_prompts.py} and \texttt{jadx\_prompts.py} hold tool-specific instructions for evidence retrieval (symbol lookup, minimal hunk extraction, JNI boundary cues). \texttt{vulnAssesment\_prompts.py} encodes the triage schema and the evidence-anchoring style used to elicit precise, verifiable rationales from the \gls{llm}.

\paragraph{Evaluation harness.}
\texttt{evaluation.py} loads saved judgements and computes summary statistics or exports artefacts for manual review; it is deliberately decoupled from the online agent to support offline auditing and ablations.



\begin{figure}
    \centering
    \scalebox{0.8}{  \begin{forest}
    for tree={font=\sffamily, grow'=0,
    folder indent=.9em, folder icons,
    edge=densely dotted}
    [AutomaticAnalysis, is root
      [CrashSummary.py, is file]
      [evaluation.py, is file]
      [ghidraMCP\_helper\_functions.py, is file]
      [jadx\_helper\_functions.py, is file]
      [main.py, is file]
      [utils.py, is file]
      [ghidra-cli, is file]
      [MCPs
        [geminiCLI.py, is file]
        [get\_agent.py, is file]
        [ghidraMCP.py, is file]
        [jadxMCP.py, is file]
        [vulnAssessment.py, is file]
        [prompts
          [ghidra\_prompts.py, is file]
          [jadx\_prompts.py, is file]
          [vulnAssesment\_prompts.py, is file]
        ]
      ]
]
  \end{forest}}
    \caption{Program Structure}
    \label{fig:program-structure}
\end{figure}

\textcolor{red}{DA AGGIORNARE}

\section{\gls{llm} and \gls{mcp} integration}
The system is \gls{llm}-agnostic, it is not tied to a single model but supports multiple options. This flexibility allows selection of the provider that best fits the task, easy switching between models, and integration of custom or domain-specific \gls{llm}s when required. Currently, the program supports \texttt{OpenAI}, \texttt{Gemini} and local \texttt{Ollama} models, and can be extended to include additional vendors.


The system integrates Large Language Models (LLMs) with reverse-engineering tools through the Model Context Protocol (MCP). This combination enables tool-grounded reasoning, where the LLM can access, query, and interpret program artefacts in a controlled and reproducible manner.  
The implementation follows a modular and provider-agnostic architecture, ensuring that the same triage logic can operate with different LLM vendors or even fully local models. This flexibility simplifies experimentation and allows future substitution of models without altering the rest of the pipeline.


%\textcolor{red}{TO add introfuction}

\subsection{Agent setup}
The agent is created through the function \texttt{get\_agent} defined in \texttt{MCPs/get\_agent.py}.  
This function instantiates a \texttt{pydantic\_ai.Agent} object that coordinates the chosen \gls{llm} and a list of MCP toolsets (\texttt{MCPServerStdio}).  
The agent receives:
\begin{itemize}
    \item a \emph{system prompt}, which defines its high-level reasoning role (for example, vulnerability assessor);
    \item an \emph{output type}, specified as a strict \texttt{pydantic} model that constrains the expected JSON schema of responses;
    \item a list of connected MCP servers.
\end{itemize}

The function automatically selects the appropriate model provider according to the prefix of the model name supplied at start-up:
\begin{itemize}
    \item names beginning with \texttt{gpt-} initialise an \texttt{OpenAIChatModel} via the \texttt{OpenAIProvider};
    \item names beginning with \texttt{gemini-} initialise a \texttt{GoogleModel} via the \texttt{GoogleProvider};
    \item any other identifier defaults to a locally hosted model served through the \texttt{OllamaProvider}.
\end{itemize}
This logic allows seamless switching between cloud-based and local models.


\paragraph{Gemini-CLI.}
If the model name is specified as \texttt{gemini-cli}, the program interacts with the Gemini Command-Line Interface instead of the standard \gls{api} endpoint.
This configuration was introduced to bypass \gls{api} usage quotas and benefit from the higher request limits available with \texttt{gemini-2.5-pro}.

The integration is handled by the helper module \texttt{MCPs/geminiCLI.py}, which provides two key functions: \texttt{query\_gemini\_cli()} and \texttt{gemini\_response\_parser()}.
Since the Gemini CLI does not support distinct system and user prompts or structured JSON output (as required by \texttt{pydantic\_ai.Agent}), these functions implement a custom prompting and parsing layer.

\texttt{query\_gemini\_cli()} builds a unified text prompt combining the system and user contexts, executes the \texttt{gemini} command via subprocess, and retries up to four times if parsing fails, Listing~\ref{lst:geminiCliPrompr}.
The model’s response is then passed to \texttt{gemini\_response\_parser()}, which cleans the raw \gls{cli} output and attempts to parse the result as JSON. If the output cannot be validated against the expected schema, the function retries; after four failed attempts, an error is raised.

\begin{lstlisting}[language=pyPrompt, caption={Prompt used in Gemini-CLI. The \texttt{system\_prompt} and \texttt{user\_prompt} are identical to those used in the standard cloud-based \gls{llm} agent, while \texttt{require\_response} defines the JSON structure that the model must produce.}
, label={lst:geminiCliPrompr}]
IMPORTANT: DO NOT MODIFY OR EXECUTE CODE. RESPOND ONLY VIA MCP CALLS. DO NOT ANALYZE OR RUN FILES. READ-ONLY CONTEXT.

SYSTEM PROMPT (upstream, for reference — do not override constraints above):
{system_prompt}

USER PROMPT: {user_prompt}

RESPONSE INSTRUCTIONS:
1) Use MCP tools/servers only if needed to answer.
2) Do not edit or run any code.
3) If the user asks for code changes/execution, refuse and propose an MCP-only path.

{response_str}

\end{lstlisting}


This approach ensures compatibility between the CLI model and the structured output format required by the pipeline, enabling the program to use \texttt{gemini-2.5-pro} locally while maintaining schema-validated results consistent with API-based executions.


%If the model name is specified as \texttt{gemini-cli}, the program interacts with the Gemini Command-Line Interface rather than the API endpoint.
%The configuration with \texttt{gemini-cli} was added since it allow higher limits of use about \texttt{gemini-2.5-pro} without directly use the \glspl{api}. The use of gemini-cli along other normal cloud \gls{api} models required extra configurations, infanct exist the helper function file \texttt{MCPs/geminiCLI.py}, that provied `query\_gemini\_cli`, with a sligtly edited propt to shere to the \gls{llm}, infact gemini cli doesn't have the system prompt and user prompt, have just the standart prompt, and also gemini cli doesn't support \texttt{pydantic\_ai.Agent} and the required JSON output, so I have to manage it manualy, and verifing the correct structure that I  required using `gemini_response_parser`, thet returns the output that I wanted, the model have 4 atepts to create that output, otherwise the program reise an error



\paragraph{Local LLM (Ollama)}
If the model name does not match either of the previous prefixes (\texttt{gpt-}/\texttt{gemini-}), the agent assumes a local configuration and instantiates an \texttt{OpenAIChatModel} paired with the \texttt{OllamaProvider}.  
This setup enables on-premise execution and full data isolation, making it suitable for offline research or restricted environments where cloud connectivity or data sharing is limited.

The main limitation of offline models is computational power.  
Local inference lacks the large-scale infrastructure available to cloud-based providers: even high-end consumer hardware struggles to run models exceeding 20~billion parameters, which typically require around 14~GB of GPU VRAM.  
While larger models exist, they demand specialised GPUs and significant memory resources.  

Among the locally tested models, \texttt{gpt-oss:20b}\footnote{\url{https://ollama.com/library/gpt-oss}} provided the best compatibility with \glspl{mcp} and \texttt{pydantic\_ai} type casting.  
However, it still showed limited reliability when producing structured JSON outputs and to get the informations from \glspl{mcp}.  
Due to these constraints, the implementation ultimately switched to the \texttt{gemini-cli} configuration, which offered a better balance between accessibility, performance, and support for schema-constrained outputs.


\subsection{\gls{mcp} setup}

To enable the \gls{llm} to automatically perform triage, two \glspl{mcp} are used: one for \texttt{Ghidra} and one for \texttt{Jadx}.  
Both require additional setup to operate autonomously within the triage pipeline.

\subsubsection{Jadx}

The integration of Jadx into the pipeline relies on the \texttt{jadx-ai-mcp}\footnote{\url{https://github.com/zinja-coder/jadx-ai-mcp}} server, which exposes Jadx features through the \gls{mcp} protocol.  
Listing~\ref{lst:jadxServer} shows how the \gls{mcp} connection for Jadx is established using the \texttt{MCPServerStdio} interface, invoking the \texttt{jadx\_mcp\_server.py} script through \texttt{uv} from the directory specified in the environment variable \texttt{JADX\_MCP\_DIR}.


\begin{lstlisting}[language=pyMCP, caption={Function \texttt{make\_jadx\_server} used to initialise the Jadx MCP connection}, label={lst:jadxServer}]
def make_jadx_server(timeout: int = 60) -> MCPServerStdio:

    jadx_dir = os.getenv("JADX_MCP_DIR")
    return MCPServerStdio(
        "uv",
        args=["--directory", jadx_dir, "run", "jadx_mcp_server.py"],
        timeout=timeout,
    )
\end{lstlisting}



This server depends on the graphical Jadx application (\texttt{jadx-gui}) being active.  
Therefore, when analysis of an \gls{apk} begins, the program launches a dedicated Jadx session using the command: \verb|jadx-gui <apk>| 
This opens the \gls{gui} instance and allows the \texttt{jadx-ai-mcp} plugin to start its server.  

Session management is handled by the helper module \texttt{jadx\_helper\_functions.py}, which contains utilities to start, monitor, and terminate the Jadx process.  
The function \texttt{start\_jadx\_gui()} launches the GUI, monitors its output stream, and blocks execution until the MCP server is detected or a timeout occurs.  
If Jadx fails to start within the configured timeout, the function terminates the process and aborts the analysis.  
The module provides \texttt{kill\_jadx()}, used to gracefully stop the Jadx process when it is no longer required or when switching to a different \gls{apk}.  
Signal handlers ensure that Jadx is automatically terminated if the main program exits unexpectedly.

Once the server is running, the triage agent connects to the \texttt{jadx-ai-mcp} endpoint to query project-level information.  
Listing~\ref{lst:getMetadata} shows how the system retrieves the \gls{apk} metadata (application name, package, minimum and target SDK, and version) using a dedicated agent call.

\begin{lstlisting}[language=pyMCP, caption={Method \texttt{get\_jadx\_metadata} that returns the app metadata from Jadx}, label={lst:getMetadata}]
server: MCPServerStdio = make_jadx_server()
...
async with get_agent(JADX_APP_METADATA, AppMetadata, [server], model_name=model_name) as j_agent:
    j_meta = await j_agent.run("Extract app metadata from the currently open Jadx project.")
appMetadata: AppMetadata = j_meta.output   
...
\end{lstlisting}

\paragraph{\texttt{jadx-ai-mcp} tools.}
The \gls{mcp} server exposes a set of structured tools that allow the agent to query, explore, and manipulate the currently opened Jadx project.  
The tools provaided are:
\begin{myitemize}
  \item \texttt{get\_android\_manifest}, \texttt{get\_main\_activity\_class}, \\\texttt{get\_main\_application\_classes\_names}/\texttt{\_code}: recover app entry points and context.
  \item \texttt{get\_all\_classes}, \texttt{get\_class\_source}, \texttt{get\_methods\_of\_class}, \texttt{get\_method\_by\_name}: class/method discovery and source retrieval.
  \item \texttt{search\_method\_by\_name}, \texttt{get\_smali\_of\_class}: locate JNI bridges and low-level details.
  \item \texttt{get\_all\_resource\_file\_names}, \texttt{get\_resource\_file}, \texttt{get\_strings}: resource inspection for configuration-driven behaviours.
  \item \texttt{fetch\_current\_class}, \texttt{get\_selected\_text}: GUI-assisted context capture during investigations.
  \item \texttt{rename\_class}/\texttt{rename\_method}/\texttt{rename\_field}: optional refactoring to improve readability (kept out of formal evidence).
\end{myitemize}

\subsubsection{Ghidra}
The integration of Ghidra into the pipeline relies on the \texttt{GhidraMCP}\footnote{\url{https://github.com/LaurieWired/GhidraMCP}}, which exposes Ghidra’s analysis and decompilation capabilities through the \gls{mcp} protocol. Listing~\ref{lst:ghidraServer} shows how the \gls{mcp} connection is established using the \texttt{MCPServerStdio} interface.

\begin{lstlisting}[language=pyMCP, caption={Function \texttt{make\_ghidra\_server} used to initialise the Ghidra MCP connection}, label={lst:ghidraServer}]
def make_ghidra_server(debug: bool = False, verbose: bool = False, timeout: int = 120) -> MCPServerStdio:
    ghidra_mcp_dir = os.getenv("GHIDRA_MCP_DIR")
    ghidra_dir = os.getenv("GHIDRA_INSTALL_DIR")
        
    return MCPServerStdio(
        "python3",
        args=[
            f"{ghidra_mcp_dir}/bridge_mcp_ghidra.py",
            "--ghidra-server", "http://127.0.0.1:8080/",
        ],
        env={"GHIDRA_INSTALL_DIR": ghidra_dir},
        timeout=timeout,
    )
\end{lstlisting}

Unlike Jadx, Ghidra does not natively support project creation and binary import directly from the terminal, which complicates its automation.  
To overcome this limitation, the pipeline employs \texttt{ghidra-cli}\footnote{\url{https://github.com/Denloob/ghidra-cli}}, a command-line utility that allows project creation and binary import using simple commands such as: \verb|ghidra-cli -n -i ./|
This command launches the Ghidra \gls{gui} (Figure~\ref{fig:ghidraGUI}) and automatically imports the specified ELF binaries (\texttt{.so} files extracted from the target \gls{apk}) into a new project.  
However, the Ghidra \gls{mcp} server becomes available only after the binary is opened in the CodeBrowser (the main Ghidra interface, represented by the dragon icon), and this selection cannot be made through the command line.

To fully automate this step, the helper module \texttt{ghidraMCP\_helper\_functions.py} was developed.  
This module uses \texttt{wmctrl} to manage desktop windows and \texttt{pyautogui} to simulate keyboard input (e.g., \texttt{Tab}, \texttt{Down Arrow}, and \texttt{Enter}) to select and open the imported file within the GUI automatically.

%The integration of Ghidra into the pipeline relies on the \texttt{GhidraMCP}\footnote{\url{https://github.com/LaurieWired/GhidraMCP}}, which exposes Ghidra’s analysis and decompilation capabilities through the \gls{mcp} protocol.  
%Unlike Jadx, Ghidra doesn't supports the start and import of new projects in an easy way from therminal, it have some issues, a work around was to use \texttt{ghidra-cli}\footnote{\url{https://github.com/Denloob/ghidra-cli}}, That allows to easily create new project and import binary using \gls{cli}, by using the command \verb|ghidra-cli -n -i ./my_binary|, this start the Ghidra \gls{gui} (Figure~\ref{fig:ghidraGUI}), importing the ELF binaries (\texttt{.so} files extracted from the target \gls{apk}) in a random name project, the problem for here is that there isn't a cli way to select the imported file, and the Ghidra \gls{mcp} starts only when the binary is launched with che Ghidra's CodeBrowser (the dragon Icon). To Automate this operation \texttt{ghidraMCP\_helper\_functions.py} was developed, this helper modul use \texttt{wmctrl} to switch pc windows and \texttt{pyautogui} to send keyboard signal, as `tab`, `down arrow` and `enter` to select the file automaticly from cli.


\begin{figure}
    \centering
    \includegraphics[width=12cm]{figures/GhidraGUI.jpeg}
    \caption{Ghidra GUI after being launched via \texttt{ghidra-cli}.}
    \label{fig:ghidraGUI}
\end{figure}

Session management is entirely handled within \texttt{ghidraMCP\_helper\_functions.py}, which provides utilities to launch, monitor, and close Ghidra processes.  
The function \texttt{openGhidraGUI()} invokes \texttt{ghidra-cli} to initialise the environment, ensures no other active Ghidra instances are running, and confirms successful startup through window detection.  
If a window titled “Ghidra:” is already open, it is automatically closed before the new session begins.  
The helper then brings the GUI to the foreground and completes the import process through simulated keyboard input.  
A built-in timeout ensures that the Ghidra environment is properly initialised before the analysis continues.


Once the \gls{gui} environment is ready, the triage agent establishes communication with the Ghidra MCP server through the \texttt{make\_ghidra\_server()} function defined in \texttt{ghidraMCP.py}.  
Listing~\ref{lst:ghidraServer}

\begin{lstlisting}[language=pyMCP, caption={Function \texttt{make\_ghidra\_server} used to initialise the Ghidra MCP connection}, label={lst:ghidraServer}]
def make_ghidra_server(debug: bool = False, verbose: bool = False, timeout: int = 120) -> MCPServerStdio:
    ghidra_mcp_dir = os.getenv("GHIDRA_MCP_DIR")
    ghidra_dir = os.getenv("GHIDRA_INSTALL_DIR")
        
    return MCPServerStdio(
        "python3",
        args=[
            f"{ghidra_mcp_dir}/bridge_mcp_ghidra.py",
            "--ghidra-server", "http://127.0.0.1:8080/",
        ],
        env={"GHIDRA_INSTALL_DIR": ghidra_dir},
        timeout=timeout,
    )
\end{lstlisting}

Listing~\ref{lst:mcpVulnAssessment} shows the use of \texttt{GhidraMCP} for perform vulnerability triage process. 

\paragraph{\texttt{GhidraMCP} tools.}
The Ghidra MCP server exposes a variety of analysis and decompilation tools accessible to the triage agent through structured \gls{mcp} calls.  
The main capabilities exposed through the server are:

\begin{myitemize}
  \item \textbf{Decompilation and Disassembly:}  
  \texttt{disassemble\_function} (return: assembly text).
  \texttt{decompile\_function}, \texttt{decompile\_function\_by\_address} (return: C code);  
  
  \item \textbf{Function and Address Queries:}  
  \texttt{get\_current\_address}, \texttt{get\_current\_function},  
  \texttt{get\_function\_by\_address} (return: name/object),  
  \texttt{get\_function\_xrefs}, \\ \texttt{get\_xrefs\_from}, \texttt{get\_xrefs\_to} (return: reference lists).
  
  \item \textbf{Listing and Enumeration:}  
  \texttt{list\_functions}, \texttt{list\_methods}, \texttt{list\_classes}, \\
  \texttt{list\_namespaces}, \texttt{list\_segments}, \texttt{list\_data\_items},  \texttt{list\_imports},  \\ 
  \texttt{list\_exports},  \texttt{list\_strings} (return: lists of names, symbols, or values).
  
  \item \textbf{Search and Cross-References:}  
  \texttt{search\_functions\_by\_name} (substring search),  \\
  \texttt{get\_function\_xrefs}, \texttt{get\_xrefs\_from}, \\
  \texttt{get\_xrefs\_to} (return: reference objects).
  
  \item \textbf{Editing and Annotation:}  \texttt{rename\_function\_by\_address}, \texttt{rename\_function}, \\
  \texttt{set\_function\_prototype}, \texttt{set\_local\_variable\_type},  \texttt{rename\_variable}, \\
\texttt{rename\_data},  \texttt{set\_decompiler\_comment}, \texttt{set\_disassembly\_comment} (return: flag or void).
\end{myitemize}












%\subsubsection{Ghidra}
%For integrate Ghidra in the tool was use \texttt{jadx-ai-mcp}\footnote{https://github.com/LaurieWired/GhidraMCP}

% \subsection{Use of Pydantic}

\section{POIROT's output and use}

As mentioned in Chapter~\ref{chp:preliminaries}, the triage procedure begins from the output generated by POIROT.  
During its \emph{Triage} phase\footnote{\url{https://github.com/HexHive/droidot/\#triage}}, POIROT produces a file named \texttt{folder2backtraces.txt}, shown in Listing~\ref{lst:POIROT-output}.  
This file serves as the primary input for the analysis performed in this project.  
After running POIROT on a batch of applications, the resulting directory structure for the target APK appears as illustrated in Figure~\ref{fig:poirot_output}.


\begin{figure}
    \centering
    \scalebox{0.8}{  \begin{forest}
    for tree={font=\sffamily, grow'=0,
    folder indent=.9em, folder icons,
    edge=densely dotted}
    [target\_APKs, is root
      [<PkgName1>
        [base.apk, is file]
        [lib]
        [fuzzing\_output/<fname-signature>/reproduced\_crashes/folder2backtraces.txt, is file]
      ]
      [<...>]
      [<PkgNameN>]
]
  \end{forest}}
    \caption{POIROT's output}
    \label{fig:poirot_output}
\end{figure}

For each package contained in \texttt{target\_APK}, the program explores the \texttt{fuzzing\_output} directory and identifies all entries named \texttt{<functionName-Signature>} that include both a \texttt{reproduced\_crashes} folder and the file \texttt{folder2backtraces.txt}.  
From this structure, the program builds a map in which the key is the package name (\texttt{pkgName}) and the value is a list of backtrace files.  
This mapping is then used to extract APK metadata and perform the vulnerability triage.

For ease of use, the program accepts the argument \texttt{--apk-list}, which specifies ``the path to a \texttt{.txt} file containing the list of APKs or application names to be included (one per line).''  
When this option is provided, only the APKs that appear in both the \texttt{.txt} file and the \texttt{target\_APK} directory are considered for triage.  
The list file also supports comments: any line starting with the character \texttt{\#} is ignored during parsing.


\section{Triaging of Vulnerabilities}
Listing~\ref{lst:mcpVulnAssessment} shows the asynchronous routine responsible for coordinating the vulnerability triage process. 
It initialises both Jadx and Ghidra \glspl{mcp}, launches the corresponding GUI sessions, and iteratively sends each crash report to the analysis agent for structured assessment.

\begin{lstlisting}[language=Python, caption={Asynchronous function \texttt{mcp\_vuln\_assessment} performing the vulnerability triage through Jadx and Ghidra MCP servers}, label={lst:mcpVulnAssessment}]
async def mcp_vuln_assessment(
    model_name: str,
    crashes: Crashes,
    relevant_libs_map: Dict[Path, List[str]],
    timeout: int = 60,
    verbose: bool = False,
    debug: bool = False
) -> AnalysisResults:

    ghidra_server = make_ghidra_server(timeout=timeout)
    jadx_server = make_jadx_server(timeout=timeout)

    results = AnalysisResults()

    ...
    
    openGhidraGUI(sorted_libs, debug=debug)
    openGhidraFile(sorted_libs, sorted_libs[0], debug=debug)
        
    for i, crash in enumerate(crashes, start=1):
        ...
        query = (
            f"Assess the following crash and provide a vulnerability assessment "
            f"in the specified format.\n{str(crash)}\n"
            f"This is a map where each key is a Path to a relevant .so library, "
            f"and the value is the list of JNI methods it implements:\n{libs_map}"
        )
        
        async with get_agent(
            ASSESSMENT_SYSTEM_PROMPT,
            VulnAssessment,
            [jadx_server, ghidra_server],
            model_name=model_name,
        ) as agent:
            resp = await agent.run(query)
        vuln = resp.output

        results.append(AnalysisResult(crash=crash, assessment=vuln))
    
    ...

    closeGhidraGUI(debug=debug)

    return results
\end{lstlisting}


\subsection{Extraction of .so files}
POIROT already extracts the native library files as part of its output. As shown in Figure~\ref{fig:poirot_output}, the \texttt{lib/} directory contains compiled libraries organised by \glspl{abi} (e.g., \texttt{arm64-v8a}, \texttt{x86\_64}, etc.).  
The triage tool uses a preference list of \glspl{abi} to select the variant to analyse first; for example, it prioritises \texttt{arm64-v8a} since most modern Android devices support 64-bit ARM.

Using the function \texttt{find\_relevant\_libs(..)}, the program enumerates for each library the available functions via the Unix command \texttt{nm}, a utility that ``dumps the symbol table and its attributes from a binary executable file (including libraries, compiled object modules, shared-object files)''.  
The resulting symbol list is then correlated with the stack trace contained in \texttt{folder2backtraces.txt} so that only the relevant \texttt{.so} files are selected for triage, thereby avoiding the overload of passing too many unused libraries to the model for analysis.

\section{Output}

The final result of the triage pipeline is an easily readable and structured format that facilitates understanding of the error and the potential vulnerability.  

The tool leverages different models inheriting from \texttt{BaseModel}\footnote{\url{https://docs.pydantic.dev/latest/api/base_model/}} (from Pydantic). One of the primary ways to define a schema in Pydantic is via models: classes inheriting from \texttt{BaseModel} that define annotated attributes. :contentReference[oaicite:1]{index=1}  
Models can be thought of as similar to structs in languages like C, or as the schema for a single API endpoint. :contentReference[oaicite:2]{index=2}  
Unlike standard dataclasses, Pydantic models include built-in validation, serialization, and JSON schema generation. This allows the system to enforce that the \gls{llm} returns responses conforming to a specified structure and type range.

Using these models allows the pipeline to define a specific output format with explicit fields and types, and rely on Pydantic to enforce compliance with these constraints.

Listing~\ref{lst:AppMetadata} shows an example of a \texttt{BaseModel} element, namely \texttt{AppMetadata}. It stores metadata about an Android app (extracted via Jadx). The method \_\_str\_\_(..) enables easy display in the CLI when the \texttt{--verbose} option is used, and \texttt{to\_json} converts the model into a JSON string.


\begin{lstlisting}[language=Python, caption={Example of the \texttt{AppMetadata} model class}, label={lst:AppMetadata}]
class AppMetadata(BaseModel):
    app_name: str
    package: str
    min_sdk: Optional[int] = None
    target_sdk: Optional[int] = None
    version_name: Optional[str] = None
    version_code: Optional[str] = None
    
    def __str__(self) -> str:
        fields = [
            f"App Name     : {self.app_name}",
            f"Package      : {self.package}",
            f"Min SDK      : {self.min_sdk or 'N/A'}",
            f"Target SDK   : {self.target_sdk or 'N/A'}",
            f"Version Name : {self.version_name or 'N/A'}",
            f"Version Code : {self.version_code or 'N/A'}"
        ]
        return "AppMetadata : \n" + "\n".join(fields)
    
    def to_json(self, *, indent: int = 2, exclude_none: bool = True, ensure_ascii: bool = False, ) -> str:
        data = self.model_dump(exclude_none=exclude_none)
        return json.dumps(data, indent=indent, ensure_ascii=ensure_ascii)
\end{lstlisting}

Some of the principal \texttt{BaseModel} classes used in the pipeline include:
\begin{itemize}
    \item \texttt{ToolInfo}: Information about the tool and environment used for the detection.
    \item \texttt{AppMetadata}: Metadata about an Android app extracted from Jadx.
    \item \texttt{AnalysisBlock}: Contains the full analysis context, including \texttt{ToolInfo}, \texttt{AppMetadata}, and \texttt{AnalysisResults}.
    \item \texttt{VulnDetection}: Result of a vulnerability assessment for a single crash.
    \item \texttt{AnalysisResult}: Combines a \texttt{CrashSummary} with its corresponding \texttt{VulnDetection}.
\end{itemize}

The use of \texttt{BaseModel} ensures both the validation of outputs returned by the \gls{llm} and straightforward conversion to a standard JSON format.

\subsection*{JSON Output}

Listing~\ref{lst:jsonOutput} shows how the json looks like. \textcolor{red}{TO FIX}





\section{How to use}
This section describes the complete workflow for running the triage pipeline—from preparing the POIROT output and configuring the environment, to launching the program and interpreting the generated results.


\subsection{Docker}\label{chp:docker}

This project adopted Docker to encapsulate the entire triage system and its dependencies in a reproducible and portable image. Docker is an open-source platform designed to automate the deployment of applications within lightweight, isolated environments known as \emph{containers}. Each container bundles an application together with its dependencies, configuration files and operating system libraries, ensuring consistent behaviour across different environments. 

This ensures that the analysis pipeline can be executed identically on local workstations, cloud servers or continuous integration infrastructures, regardless of the host configuration.

The \texttt{Dockerfile} defines a minimal image that includes all components required:
\begin{myitemize}
    \item Python~3.12 runtime with all dependencies;
    \item the \texttt{pydantic} and \texttt{pydantic\_ai} libraries for typed models and agent orchestration;
    \item \texttt{Ghidra} and \texttt{Jadx} and reciprocal \gls{mcp} servers;
    \item environment configuration for \texttt{LLM\_MODEL\_NAME}, \texttt{GHIDRA\_MCP\_DIR}, and \texttt{JADX\_MCP\_DIR}.
\end{myitemize}

\vspace{4mm}

Two auxiliary scripts simplify the deployment and execution of the containerised system.  
The first, \texttt{dockerSetup.sh}, automates the build process and initial execution of the container. It builds the container from the provided \texttt{Dockerfile} and assigns it the name \texttt{triage\_dev}. By naming the container, its configuration and state are retained across sessions, allowing the environment to be restarted without needing to reconfig it each time.

During setup, \texttt{dockerSetup.sh} mounts two persistent volumes:
\begin{itemize}
    \item \textbf{workspace}: synchronises the project’s source code between the host machine and the container, enabling code edits and updates to take effect immediately without rebuilding the image;
    \item \textbf{APKs}: provides shared access to the APK files to be analysed. If the APKs (the POIROT’s output) are placed in this directory, the container can access them directly; otherwise, the container remains isolated from the host filesystem, ensuring safe separation of data.
\end{itemize}

The first execution of the container requires several additional configuration steps to enable the integration of \gls{mcp} servers and authentication for Gemini-CLI (if used):

\paragraph{Gemini CLI setup.}
Gemini authentication must be performed once, as the CLI requires an OAuth credential file. Since the Docker container lacks browser access, the login is completed externally on the host machine. 

\paragraph{Jadx MCP server.}
The Jadx MCP server is generally installed automatically when the container is first built. However, if errors occur, it can be installed manually by following the instructions.

\paragraph{Ghidra MCP server.}
The Ghidra MCP server is pre-packaged in the Docker image, but not installed, since required the use of the \gls{gui}. To activate it, Ghidra must first be launched in the container using. The setup then follows the official configuration steps described in the GhidraMCP documentation.

The second script, \texttt{dockerRun.sh}, provides a simplified interface for launching the container after it has been configured. It executes only a few lines to start the existing \texttt{triage\_dev} container, resuming the previously configured environment without rebuilding or reinitialising it.


\subsection{Program arguments}

The program is executed from the command line using Python~3:

\begin{verbatim}
python3 main.py <target_APK> [options]
\end{verbatim}

where \texttt{<target\_APK>} is the path to the POIROT output directory containing the analysed applications.  
The command-line arguments control which APKs are processed, which model and timeout are used, and where results are stored.

\textbf{Positional argument}
    \begin{myitemize}
      \item \texttt{target\_APK}: Path to the POIROT output folder (containing one subdirectory per analysed application, e.g.\ \texttt{PkgName/}).
    \end{myitemize}

\textbf{Optional arguments}
    \begin{myitemize}
      \item \texttt{--apk-list}~\textit{Path}: Path to a \texttt{.txt} file containing the list of APKs or application names to include in the triage (one per line).  
      Lines beginning with \texttt{\#} are treated as comments and ignored.  
      Only APKs present in both the directory and this list are analysed.
      
      \item \texttt{-m, --model-name}~\textit{Str}: Specifies the \gls{llm} model name to use for the triage.  
      Defaults to the value of the environment variable \texttt{LLM\_MODEL\_NAME} or \texttt{gemini-2.5-flash} if not set.
      
      \item \texttt{-o, --out-dir}~\textit{Path}: Base directory where reports are saved.  
      If omitted, a directory named \texttt{classification\_YYYY\_MM\_DD\_HH:MM} is automatically created.
      
      \item \texttt{--timeout}~\textit{Int}: Timeout in seconds for MCP server operations (default: 180\,s).
      
      \item \texttt{-d, --debug}: Enables detailed debug logs.  
      Prints diagnostic information, subprocess output, and environment variables used for configuration.
      
      \item \texttt{-v, --verbose}: Echoes system and user prompts exchanged with the \gls{llm}, as well as the model’s intermediate reasoning.
    \end{myitemize}

\paragraph{Example.}
\begin{verbatim}
python3 main.py ./target_APK \
    --apk-list ./apk_list.txt \
    -m gemini-2.5-flash \
    -o ./results \
    --timeout 240 \
    --debug
\end{verbatim}

This command analyses all APKs listed in \texttt{apk\_list.txt} found inside \texttt{./target\_APK},  
uses the model \texttt{gemini-2.5-flash}, saves the reports in \texttt{./results},  
sets a timeout of 240 seconds for MCP servers, and enables debug output.