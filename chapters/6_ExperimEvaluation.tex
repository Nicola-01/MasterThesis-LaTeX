\chapter{Evaluation}
\label{chp:eval}

\section{Experimental Setup}

The evaluation of the tool was performed on a local machine, using the Docker-based environment described in Chapter~\ref{chp:docker}.  
The environment consisted of:

\begin{itemize}
    \item Local machine running \textbf{Ubuntu 25.04};
    \item Docker container based on \textbf{Ubuntu 24.04};
    \item \textbf{Ghidra~11.4.2} (latest version at the time of evaluation);
    \item \textbf{Jadx~1.5.3} (latest version at the time of evaluation);
    \item \textbf{GhidraMCP~1.4} (latest version at the time of evaluation);
    \item \textbf{jadx-ai-mcp~4.0.0} (latest version at the time of evaluation).
\end{itemize}

\subsection{AWS for POIROT}

The extraction of crashes used for this evaluation is described in Chapter~\ref{chp:preliminaries}, and the corresponding execution environment is detailed in Chapter~\ref{chp:preliminaries_env}.  
All POIROT runs were executed on an AWS EC2 instance configured for large-scale fuzzing.

\subsection{LLM Used}

The evaluation was conducted using the model \textcolor{red}{OpenAI GPT-5}.

\section{Test Applications}

Table~\ref{tab:apps} lists the applications used for the evaluation, together with the exact versions of each APK.  
These apps were selected from the POIROT dataset and filtered based on the run performed. %The applications may have different versions in the POIROT dataset and this dataset. \textcolor{red}{Non so se tenere questa ultima frase per sengalare che le apks hanno versioni diverse}

\begin{table}[ht]
\centering
\resizebox{1\textwidth}{!}{
\begin{tabular}{|l|l||l|l|}
\hline
\multicolumn{4}{|c|}{\textbf{Applications Used for Evaluation}} \\
\hline
\textbf{App} & \textbf{Version} & \textbf{App} & \textbf{Version} \\
\hline
br.com.pedidos10 & 1.16.4 & ca.radioplayer.android & 6.3.420.1 \\
com.ahnlab.v3mobileplus & 2.5.20.10 & com.appgeneration.itunerfree & 9.3.13 \\
com.cisco.webex.meetings & 45.3.0 & com.clearchannel.iheartradio.controller & 10.36.0 \\
com.ford.fordpasseu & 4.23.1 & com.hyundaicard.cultureapp & 1.0.72 \\
com.intsig.BCRLite & 7.85.5 & com.jeju.genie & 2.2.13 \\
com.kbstar.kbbridge & 1.2.1 & com.kt.ktauth & 02.01.37 \\
com.lottemembers.android & 7.7.5 & com.more.dayzsurvival.gp & 25.1002.1 \\
com.pandora.android & 2509.1 & com.rockbite.deeptown & 6.2.10 \\
com.rstgames.durak & 1.9.11 & com.samsungcard.shopping & 1.4.901 \\
com.skmc.okcashbag.home\_google & 7.0.8 & com.skt.prod.dialer & 13.6.5 \\
com.skt.smartbill & 6.4.1 & com.sopheos.videgreniersmobile & 20.11.10 \\
com.ssg.serviceapp.android.egiftcertificate & 2.6.10 & com.teamjin.deliveryk & 7.0.3 \\
com.tencent.mm & 8.0.28 & com.tplink.skylight & 3.1.20 \\
fr.radioplayer.android & 6.6.420.1 & kr.co.busanbank.mbp & 3.0.10 \\
kr.co.morpheus.geps & 02.42 & kr.co.samsungcard.mpocket & 5.4.306 \\
kr.go.iros & 1.2.3 & kr.go.kcs.mobile.pubservice & 1.0.281 \\
kr.go.minwon.m & 2.5.95 & kr.go.nts.android & 12.8 \\
kr.go.wetax.android & 5.4.14 & nh.smart.allonebank & 1.8.8 \\
nh.smart.banking & 4.1.2 & nh.smart.card & 6.5.0 \\
nh.smart.nhcok & 2.0.70 & ragazzo.alphacode.com.br & 3.0.9 \\
tw.com.taishinbank.ccapp & 5.631 & tw.gov.tra.twtraffic & 2.1.2 \\
\hline
\end{tabular}
}
\caption{Applications evaluated by the triage system.}
\label{tab:apps}
\end{table}

\textcolor{red}{Devo scrivere quali sono lastversion e quali quelle di circa 2023? }

\section{Execution Time}

Table~\ref{tab:apps_time} reports the execution time for each application.  
The total time per APK depends primarily on the \emph{number of crashes} produced by POIROT, as well as the number of relevant libraries and the complexity of the stack trace.

\begin{table}[ht]
\centering
\resizebox{1\textwidth}{!}{
\begin{tabular}{|l|l||l|l|}
\hline
\multicolumn{4}{|c|}{\textbf{Execution Time per Application}} \\
\hline
\textbf{App} & \textbf{Time (s)} & \textbf{App} & \textbf{Time (s)} \\
\hline
br.com.pedidos10 & TIME & ca.radioplayer.android & TIME \\
com.ahnlab.v3mobileplus & TIME & com.appgeneration.itunerfree & TIME \\
com.cisco.webex.meetings & TIME & com.clearchannel.iheartradio.controller & TIME \\
com.ford.fordpasseu & TIME & com.hyundaicard.cultureapp & TIME \\
com.intsig.BCRLite & TIME & com.jeju.genie & TIME \\
com.kbstar.kbbridge & TIME & com.kt.ktauth & TIME \\
com.lottemembers.android & TIME & com.more.dayzsurvival.gp & TIME \\
com.pandora.android & TIME & com.rockbite.deeptown & TIME \\
com.rstgames.durak & TIME & com.samsungcard.shopping & TIME \\
com.skmc.okcashbag.home\_google & TIME & com.skt.prod.dialer & TIME \\
com.skt.smartbill & TIME & com.sopheos.videgreniersmobile & TIME \\
com.ssg.serviceapp.android.egiftcertificate & TIME & com.teamjin.deliveryk & TIME \\
com.tencent.mm & TIME & com.tplink.skylight & TIME \\
fr.radioplayer.android & TIME & kr.co.busanbank.mbp & TIME \\
kr.co.morpheus.geps & TIME & kr.co.samsungcard.mpocket & TIME \\
kr.go.iros & TIME & kr.go.kcs.mobile.pubservice & TIME \\
kr.go.minwon.m & TIME & kr.go.nts.android & TIME \\
kr.go.wetax.android & TIME & nh.smart.allonebank & TIME \\
nh.smart.banking & TIME & nh.smart.card & TIME \\
nh.smart.nhcok & TIME & ragazzo.alphacode.com.br & TIME \\
tw.com.taishinbank.ccapp & TIME & tw.gov.tra.twtraffic & TIME \\
\hline
\hline
\multicolumn{2}{|r||}{\textbf{Average Time}} & \multicolumn{2}{l|}{\textbf{AVG}} \\
\hline
\end{tabular}
}
\caption{Approximate execution time per application.}
\label{tab:apps_time}
\end{table}

\section{\gls{llm} Usage}

\glspl{llm}, when accessed through their cloud \glspl{api}, compute usage costs based on the number of \emph{input tokens} (the prompt sent to the model) and \emph{output tokens} (the generated response).  
Token consumption plays a central role in both performance and cost, and is influenced by several factors:


Token usage depends mainly on:

\begin{itemize}
    \item The prompt used, including both the system prompt and the user prompt;
    \item The number of crashes found by POIROT;
    \item The amount of decompiled code retrieved from Ghidra, which depends on the depth of the crash stack trace;
    \item The verbosity and structure of the triage explanation produced by the \gls{llm}.
\end{itemize}

Table~\ref{tab:apps_tokens} reports input and output token usage for each application.

\begin{table}[ht]
\centering
\resizebox{1\textwidth}{!}{
\begin{tabular}{|l|l|l||l|l|l|}
\hline
\multicolumn{6}{|c|}{\textbf{Token Usage per Application}} \\
\hline
\textbf{App} & \textbf{Input} & \textbf{Output} &
\textbf{App} & \textbf{Input} & \textbf{Output} \\
\hline
br.com.pedidos10 & NTokens & NTokens & ca.radioplayer.android & NTokens & NTokens \\
com.ahnlab.v3mobileplus & NTokens & NTokens & com.appgeneration.itunerfree & NTokens & NTokens \\
com.cisco.webex.meetings & NTokens & NTokens & com.clearchannel.iheartradio.controller & NTokens & NTokens \\
com.ford.fordpasseu & NTokens & NTokens & com.hyundaicard.cultureapp & NTokens & NTokens \\
com.intsig.BCRLite & NTokens & NTokens & com.jeju.genie & NTokens & NTokens \\
com.kbstar.kbbridge & NTokens & NTokens & com.kt.ktauth & NTokens & NTokens \\
com.lottemembers.android & NTokens & NTokens & com.more.dayzsurvival.gp & NTokens & NTokens \\
com.pandora.android & NTokens & NTokens & com.rockbite.deeptown & NTokens & NTokens \\
com.rstgames.durak & NTokens & NTokens & com.samsungcard.shopping & NTokens & NTokens \\
com.skmc.okcashbag.home\_google & NTokens & NTokens & com.skt.prod.dialer & NTokens & NTokens \\
com.skt.smartbill & NTokens & NTokens & com.sopheos.videgreniersmobile & NTokens & NTokens \\
com.ssg.serviceapp.android.egiftcertificate & NTokens & NTokens & com.teamjin.deliveryk & NTokens & NTokens \\
com.tencent.mm & NTokens & NTokens & com.tplink.skylight & NTokens & NTokens \\
fr.radioplayer.android & NTokens & NTokens & kr.co.busanbank.mbp & NTokens & NTokens \\
kr.co.morpheus.geps & NTokens & NTokens & kr.co.samsungcard.mpocket & NTokens & NTokens \\
kr.go.iros & NTokens & NTokens & kr.go.kcs.mobile.pubservice & NTokens & NTokens \\
kr.go.minwon.m & NTokens & NTokens & kr.go.nts.android & NTokens & NTokens \\
kr.go.wetax.android & NTokens & NTokens & nh.smart.allonebank & NTokens & NTokens \\
nh.smart.banking & NTokens & NTokens & nh.smart.card & NTokens & NTokens \\
nh.smart.nhcok & NTokens & NTokens & ragazzo.alphacode.com.br & NTokens & NTokens \\
tw.com.taishinbank.ccapp & NTokens & NTokens & tw.gov.tra.twtraffic & NTokens & NTokens \\
\hline
\hline
\multicolumn{3}{|r||}{\textbf{Average Input Tokens}} &
\multicolumn{3}{l|}{\textbf{AVG}} \\
\multicolumn{3}{|r||}{\textbf{Average Output Tokens}} &
\multicolumn{3}{l|}{\textbf{AVG}} \\
\hline
\end{tabular}
}
\caption{Input and output token usage per application.}
\label{tab:apps_tokens}
\end{table}

\subsubsection{Evaluation Metrics}

\textcolor{red}{DA VEDERE se serve specificare le metriche}

For the evaluation, the F1-score was selected as the primary metric. This measure is commonly employed in classification settings because it offers a single value that reflects the trade-off between precision and recall, which is especially important when the distribution of classes is unbalanced or when both false positives and false negatives carry comparable weight. The F1-score is defined as:
\[
\mathrm{F1} = 2 \cdot \frac{\text{precision} \cdot \text{recall}}{\text{precision} + \text{recall}}
\]

Precision quantifies the proportion of correctly identified positive cases among all cases predicted as positive, whereas recall quantifies the proportion of actual positive cases that were successfully detected.  
The F1-score combines these two aspects through their harmonic mean, ensuring a lower score when precision and recall diverge significantly, even if one of the two is high.
