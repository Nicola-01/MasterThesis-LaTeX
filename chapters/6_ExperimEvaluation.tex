\chapter{Evaluation}
\label{chp:eval}

This chapter evaluates the effectiveness, reliability, and overall behaviour of the proposed triage system.  
The goal is to assess how well the \gls{llm}-based approach classifies crashes and how the availability of additional program context, i.e. \gls{jcg}, influences its decisions.  
The chapter presents the experimental setup, the selected test applications, and a quantitative evaluation based on standard classification metrics.  
The results provide insight into the strengths and limitations of the current implementation and form the basis for the discussion in the following chapter.


\section{Experimental Setup}

%, using the Docker-based environment described in Chapter~\ref{chp:docker}.

The evaluation of the tool was carried out on a local machine.  
The experimental environment consisted of:

\begin{myitemize}
    \item Local machine running \textbf{Ubuntu 25.04};
    %\item Docker container based on \textbf{Ubuntu 24.04};
    \item \textbf{Ghidra~11.4.2} (latest version at the time of evaluation);
    \item \textbf{Jadx~1.5.3} (latest version at the time of evaluation);
    \item \textbf{GhidraMCP~1.4} (latest version at the time of evaluation);
    \item \textbf{jadx-ai-mcp~4.0.0} (latest version at the time of evaluation).
\end{myitemize}

\subsection{AWS for POIROT}

The extraction of crashes used for this evaluation is described in Chapter~\ref{chp:preliminaries}, and the corresponding execution environment is detailed in Chapter~\ref{chp:preliminaries_env}.  
All POIROT runs were executed on an AWS EC2 instance configured for large-scale fuzzing.

\subsection{\glsxtrshort{llm} Used}

The evaluation was conducted using the OpenAI model GPT-5.1.  
This model represented one of the most capable \glspl{llm} available at the time of testing and offered reliable support for \gls{mcp}-based tool use.
GPT-5.1 also integrates smoothly with the Pydantic framework, allowing strict control over structured outputs.

% No fine-tuning or additional training was performed, as the goal of the evaluation was to assess how a general-purpose model behaves when equipped with structured context and targeted guidance through MCP-based evidence retrieval.


\section{Test Applications}

Table~\ref{tab:apps} lists the applications used for the evaluation, together with the exact versions of each APK.  
These apps were selected from the POIROT dataset and filtered based on the run performed. %The applications may have different versions in the POIROT dataset and this dataset. \textcolor{red}{Non so se tenere questa ultima frase per sengalare che le apks hanno versioni diverse}

\begin{table}[ht]
\centering
\resizebox{1\textwidth}{!}{
\begin{tabular}{|l|l||l|l|}
\hline
%\multicolumn{4}{|c|}{\textbf{Applications Used for Evaluation}} \\
%\hline
\textbf{App} & \textbf{Version} & \textbf{App} & \textbf{Version} \\
\hline
br.com.pedidos10 & 1.16.4 & ca.radioplayer.android & 6.3.420.1 \\
com.ahnlab.v3mobileplus & 2.5.20.10 & com.appgeneration.itunerfree & 9.3.13 \\
com.cisco.webex.meetings & 45.3.0 & com.clearchannel.iheartradio.controller & 10.36.0 \\
com.ford.fordpasseu & 4.23.1 & com.hyundaicard.cultureapp & 1.0.72 \\
com.intsig.BCRLite & 7.85.5 & com.jeju.genie & 2.2.13 \\
com.kbstar.kbbridge & 1.2.1 & com.kt.ktauth & 02.01.37 \\
com.lottemembers.android & 7.7.5 & com.more.dayzsurvival.gp & 25.1002.1 \\
com.pandora.android & 2509.1 & com.rockbite.deeptown & 6.2.10 \\
com.rstgames.durak & 1.9.11 & com.samsungcard.shopping & 1.4.901 \\
com.skmc.okcashbag.home\_google & 7.0.8 & com.skt.prod.dialer & 13.6.5 \\
com.skt.smartbill & 6.4.1 & com.sopheos.videgreniersmobile & 20.11.10 \\
com.ssg.serviceapp.android.egiftcertificate & 2.6.10 & com.teamjin.deliveryk & 7.0.3 \\
com.tencent.mm & 8.0.28 & com.tplink.skylight & 3.1.20 \\
fr.radioplayer.android & 6.6.420.1 & kr.co.busanbank.mbp & 3.0.10 \\
kr.co.morpheus.geps & 02.42 & kr.co.samsungcard.mpocket & 5.4.306 \\
kr.go.iros & 1.2.3 & kr.go.kcs.mobile.pubservice & 1.0.281 \\
kr.go.minwon.m & 2.5.95 & kr.go.nts.android & 12.8 \\
kr.go.wetax.android & 5.4.14 & nh.smart.allonebank & 1.8.8 \\
nh.smart.banking & 4.1.2 & nh.smart.card & 6.5.0 \\
nh.smart.nhcok & 2.0.70 & ragazzo.alphacode.com.br & 3.0.9 \\
tw.com.taishinbank.ccapp & 5.631 & tw.gov.tra.twtraffic & 2.1.2 \\
\hline
\end{tabular}
}
\caption{Applications evaluated by the triage system.}
\label{tab:apps}
\end{table}

Across all these applications, a total of 59 methods were identified, each associated with one or more crashes.  
%In total, the dataset contained 134 crash instances.

\subsection{Execution Time}

Table~\ref{tab:apps_time} reports the \gls{llm}-analysis time for each application.  
The total time per APK depends primarily on the \emph{number of crashes} produced by POIROT, as well as the number of relevant libraries and the complexity of the stack trace.
Note that the reported time covers only the \gls{llm} classification phase and does not include the startup time of Ghidra, which depends solely on the number and size of the native libraries being imported.

\begin{table}[ht]
\centering
\resizebox{1\textwidth}{!}{
\begin{tabular}{|l|l||l|l|}
\hline
%\multicolumn{4}{|c|}{\textbf{Execution LLM Time per Application}} \\
%\hline
\textbf{App} & \textbf{Time (s)} & \textbf{App} & \textbf{Time (s)} \\
\hline
br.com.pedidos10 & 80 & ca.radioplayer.android & 480 \\ 
com.ahnlab.v3mobileplus & 338 & com.appgeneration.itunerfree & 113 \\ 
com.cisco.webex.meetings & 129 & com.clearchannel.iheartradio.controller & 200 \\ 
com.ford.fordpasseu & 185 & com.hyundaicard.cultureapp & 48 \\ 
com.intsig.BCRLite & 59 & com.jeju.genie & 33 \\ 
com.kbstar.kbbridge & 35 & com.kt.ktauth & 28 \\ 
com.lottemembers.android & 71 & com.more.dayzsurvival.gp & 25 \\ 
com.pandora.android & 420 & com.rockbite.deeptown & 17 \\ 
com.rstgames.durak & 36 & com.samsungcard.shopping & 22 \\ 
com.skmc.okcashbag.home\_google & 57 & com.skt.prod.dialer & 278 \\ 
com.skt.smartbill & 579 & com.sopheos.videgreniersmobile & 60 \\ 
com.ssg.serviceapp.android.egiftcertificate & 31 & com.teamjin.deliveryk & 83 \\ 
com.tencent.mm & 321 & com.tplink.skylight & 71 \\ 
fr.radioplayer.android & 628 & kr.co.busanbank.mbp & 805 \\ 
kr.co.morpheus.geps & 59 & kr.co.samsungcard.mpocket & 55 \\ 
kr.go.iros & 43 & kr.go.kcs.mobile.pubservice & 60 \\ 
kr.go.minwon.m & 30 & kr.go.nts.android & 87 \\ 
kr.go.wetax.android & 35 & nh.smart.allonebank & 36 \\ 
nh.smart.banking & 29 & nh.smart.card & 30 \\ 
nh.smart.nhcok & 37 & ragazzo.alphacode.com.br & 87 \\ 
tw.com.taishinbank.ccapp & 98 & tw.gov.tra.twtraffic & 78 \\ 

\hline
\hline
\multicolumn{2}{|r||}{\textbf{Average Time}} & \multicolumn{2}{l|}{\textbf{142}} \\
\hline
\end{tabular}
}
\caption{LLM analysis time per application.}
\label{tab:apps_time}
\end{table}

\subsection{Tokens usage}

\glspl{llm}, when accessed through their cloud \glspl{api}, compute usage costs based on the number of \emph{input tokens} (the prompt sent to the model) and \emph{output tokens} (the generated response).  
Token consumption plays a central role in both performance and cost.  
In this evaluation, usage varies primarily according to the size of the prompts (which include both the system and user components), the number of crashes produced by POIROT for each application, the amount of decompiled code retrieved from Ghidra—which grows with the depth and complexity of the stack trace—and the verbosity of the final triage explanation generated by the \gls{llm}.  


Table~\ref{tab:apps_tokens} reports input and output token usage for each application.

\begin{table}[ht]
\centering
\resizebox{1\textwidth}{!}{
\begin{tabular}{|l|l|l||l|l|l|}
\hline
%\multicolumn{6}{|c|}{\textbf{Token Usage per Application}} \\
%\hline
\textbf{App} & \textbf{Input} & \textbf{Output} &
\textbf{App} & \textbf{Input} & \textbf{Output} \\
\hline
br.com.pedidos10 & 36344 & 4006 & ca.radioplayer.android & 494465 & 23484 \\ 
com.ahnlab.v3mobileplus & 392719 & 17831 & com.appgeneration.itunerfree & 53776 & 6516 \\ 
com.cisco.webex.meetings & 92548 & 7643 & com.clearchannel.iheartradio.controller & 151439 & 7039 \\ 
com.ford.fordpasseu & 84360 & 6811 & com.hyundaicard.cultureapp & 14153 & 2152 \\ 
com.intsig.BCRLite & 33230 & 4095 & com.jeju.genie & 12987 & 2244 \\ 
com.kbstar.kbbridge & 13014 & 2189 & com.kt.ktauth & 41470 & 1466 \\ 
com.lottemembers.android & 38747 & 2655 & com.more.dayzsurvival.gp & 6542 & 1250 \\ 
com.pandora.android & 70891 & 9638 & com.rockbite.deeptown & 6564 & 1421 \\ 
com.rstgames.durak & 19232 & 1726 & com.samsungcard.shopping & 12380 & 1170 \\ 
com.skmc.okcashbag.home\_google & 30012 & 3708 & com.skt.prod.dialer & 212223 & 12100 \\ 
com.skt.smartbill & 359973 & 28820 & com.sopheos.videgreniersmobile & 48392 & 3213 \\ 
com.ssg.serviceapp.android.egiftcertificate & 18401 & 1734 & com.teamjin.deliveryk & 31816 & 4143 \\ 
com.tencent.mm & 877501 & 16197 & com.tplink.skylight & 29527 & 4077 \\ 
fr.radioplayer.android & 202292 & 21668 & kr.co.busanbank.mbp & 495055 & 44584 \\ 
kr.co.morpheus.geps & 22248 & 1544 & kr.co.samsungcard.mpocket & 29228 & 2915 \\ 
kr.go.iros & 24988 & 2374 & kr.go.kcs.mobile.pubservice & 25362 & 3232 \\ 
kr.go.minwon.m & 14516 & 1378 & kr.go.nts.android & 59025 & 4887 \\ 
kr.go.wetax.android & 12829 & 2305 & nh.smart.allonebank & 24960 & 1695 \\ 
nh.smart.banking & 20250 & 1610 & nh.smart.card & 14111 & 1783 \\ 
nh.smart.nhcok & 14361 & 1676 & ragazzo.alphacode.com.br & 38671 & 3792 \\ 
tw.com.taishinbank.ccapp & 31781 & 4205 & tw.gov.tra.twtraffic & 42834 & 3530 \\ 

\hline
\hline
\multicolumn{3}{|r||}{\textbf{Average Input Tokens}} &
\multicolumn{3}{l|}{\textbf{212434}} \\
\multicolumn{3}{|r||}{\textbf{Average Output Tokens}} &
\multicolumn{3}{l|}{\textbf{13963}} \\
\hline
\end{tabular}
}
\caption{Input and output token usage per application.}
\label{tab:apps_tokens}
\end{table}

\subsection{Requests and Tool Calls}

Each crash triggers a series of \emph{LLM requests}, corresponding to the number of times the agent sends a prompt to the model, and a number of \emph{LLM tool calls}, which represent successful invocations of Jadx or Ghidra through MCP during the analysis.  

Table~\ref{tab:apps_requests} reports, for each application, the number of LLM requests and MCP tool calls performed during the full triage stage.  

\begin{table}[ht]
\centering
\resizebox{1\textwidth}{!}{
\begin{tabular}{|l|l|l||l|l|l|}
\hline
%\multicolumn{6}{|c|}{\textbf{LLM request and Tool Calls per Application}} \\
%\hline
\textbf{App} & \textbf{Input} & \textbf{Output} &
\textbf{App} & \textbf{Input} & \textbf{Output} \\
\hline
br.com.pedidos10 & 5 & 9 & ca.radioplayer.android & 44 & 53 \\ 
com.ahnlab.v3mobileplus & 31 & 42 & com.appgeneration.itunerfree & 6 & 14 \\ 
com.cisco.webex.meetings & 10 & 23 & com.clearchannel.iheartradio.controller & 17 & 20 \\ 
com.ford.fordpasseu & 9 & 26 & com.hyundaicard.cultureapp & 2 & 3 \\ 
com.intsig.BCRLite & 4 & 9 & com.jeju.genie & 2 & 3 \\ 
com.kbstar.kbbridge & 2 & 3 & com.kt.ktauth & 6 & 6 \\ 
com.lottemembers.android & 2 & 5 & com.more.dayzsurvival.gp & 2 & 3 \\ 
com.pandora.android & 8 & 20 & com.rockbite.deeptown & 2 & 2 \\ 
com.rstgames.durak & 3 & 3 & com.samsungcard.shopping & 2 & 2 \\ 
com.skmc.okcashbag.home\_google & 4 & 5 & com.skt.prod.dialer & 26 & 26 \\ 
com.skt.smartbill & 51 & 82 & com.sopheos.videgreniersmobile & 6 & 10 \\ 
com.ssg.serviceapp.android.egiftcertificate & 3 & 2 & com.teamjin.deliveryk & 4 & 9 \\ 
com.tencent.mm & 49 & 42 & com.tplink.skylight & 4 & 9 \\ 
fr.radioplayer.android & 22 & 59 & kr.co.busanbank.mbp & 55 & 63 \\ 
kr.co.morpheus.geps & 3 & 4 & kr.co.samsungcard.mpocket & 4 & 8 \\ 
kr.go.iros & 4 & 4 & kr.go.kcs.mobile.pubservice & 4 & 4 \\ 
kr.go.minwon.m & 2 & 4 & kr.go.nts.android & 8 & 13 \\ 
kr.go.wetax.android & 2 & 3 & nh.smart.allonebank & 4 & 4 \\ 
nh.smart.banking & 3 & 4 & nh.smart.card & 2 & 3 \\ 
nh.smart.nhcok & 2 & 4 & ragazzo.alphacode.com.br & 5 & 13 \\ 
tw.com.taishinbank.ccapp & 4 & 10 & tw.gov.tra.twtraffic & 5 & 13 \\ 

\hline
\hline
\multicolumn{3}{|r||}{\textbf{Average LLM requests}} &
\multicolumn{3}{l|}{\textbf{16.4}} \\
\multicolumn{3}{|r||}{\textbf{Average LLM Tool Calls}} &
\multicolumn{3}{l|}{\textbf{25.3}} \\
\hline
\end{tabular}
}
\caption{LLM request and Tool Calls per application.}
\label{tab:apps_requests}
\end{table}
        
\vfill


\section{Evaluation Metrics}
%\textcolor{red}{non se serve specificare le metriche}


To objectively assess the performance of the \gls{llm}-based triage system, standard information retrieval metrics derived from a confusion matrix are employed. Given the binary nature of the classification task, determining whether a crash is a vulnerability or not, the possible outcomes are defined as follows:

\begin{itemize}
    \item \textbf{True Positive (TP):} A vulnerability correctly identified by the model. The crash represents a security risk, and the system correctly classified it as such (\texttt{is\_vulnerable: true}).
    \item \textbf{False Positive (FP):} A benign crash incorrectly flagged as a vulnerability. This represents a ''false alarm``, where the model assigns security relevance to a non-exploitable issue.
    \item \textbf{True Negative (TN):} A benign crash correctly identified as non-vulnerable. The system correctly determined that the issue does not pose a security risk.
    \item \textbf{False Negative (FN):} A vulnerability missed by the model. The crash is dangerous, but the system incorrectly classified it as benign.
\end{itemize}

Based on these definitions, the following metrics were calculated:

\paragraph{Accuracy}
Accuracy quantifies how many of the analysed crashes were assigned the correct label.  
It provides an overall view of the system’s correctness across both vulnerable and non-vulnerable cases.

\[
\text{Accuracy} = \frac{\text{TP} + \text{TN}}{\text{TP} + \text{TN} + \text{FP} + \text{FN}}
\]

\paragraph{Precision} Precision quantifies the reliability of the positive predictions. 
Indicates how often a crash flagged as vulnerable is indeed a real vulnerability.

\[
\text{Precision} = \frac{\text{TP}}{\text{TP} + \text{FP}}
\]

\paragraph{Recall}
Recall measures how effectively the system identifies all crashes that correspond to real vulnerabilities.  
Captures the model’s ability to avoid missing true security-relevant issues.  
Indicates how many actual vulnerabilities the system successfully detects.


\[
\text{Recall} = \frac{\text{TP}}{\text{TP} + \text{FN}}
\]

\paragraph{F1-Score}
The F1-score provides a single value that captures the trade-off between precision and recall.  
It reflects the balance between detecting vulnerabilities and avoiding false alarms.

\[
\mathrm{F1} = 2 \cdot \frac{\text{precision} \cdot \text{recall}}{\text{precision} + \text{recall}}
\]

\section{Results}
\label{chp:result}

The confusion matrices in Figure~\ref{fig:classification_outcomes} summarise how the system classified the 59 methods in the dataset, separating the two experimental conditions (with and without the \glsxtrlong{jcg}) and also reporting the aggregated results.

\begin{figure}
    \centering
    \begin{subfigure}{0.32\textwidth}
\scalebox{0.75}{
\begin{tabular}{c >{\bfseries}r @{\hspace{0.7em}}c 
                @{\hspace{0.4em}}c @{\hspace{0.7em}}l}
  \multirow{10}{*}{\rotatebox{90}{
      \parbox{6.2cm}{\bfseries\centering Actual\\}}} &
      & \multicolumn{2}{c}{\hspace*{-1em}\bfseries Predict} & \\
  & & \bfseries P & \bfseries N \\
  & P$'$ & \MyGradeColor{13}{21} & \MyGradeColor{3}{5} \\[2.1em]
  & N$'$ & \MyGradeColor{11}{18} & \MyGradeColor{34}{56} \\
% & P$'$ & \MyGradeColor{True}{Positive}{1} & \MyGradeColor{False}{Negative}{1} \\[2.1em]
% & N$'$ & \MyGradeColor{False}{Positive}{1} & \MyGradeColor{True}{Negative}{99} \\
\end{tabular}
}
\subcaption{Evaluation on methods that include a JCG}
\end{subfigure}
\hfill
\begin{subfigure}{0.32\textwidth}
\scalebox{0.75}{
\begin{tabular}{c >{\bfseries}r @{\hspace{0.7em}}c 
                @{\hspace{0.4em}}c @{\hspace{0.7em}}l}
  \multirow{10}{*}{\rotatebox{90}{
      \parbox{6.2cm}{\bfseries\centering Actual\\}}} &
      & \multicolumn{2}{c}{\hspace*{-1em}\bfseries Predict} & \\
  & & \bfseries P & \bfseries N \\
  & P$'$ & \MyGradeColor{10}{13} & \MyGradeColor{2}{3} \\[2.1em]
  & N$'$ & \MyGradeColor{30}{40} & \MyGradeColor{33}{44} \\
% & P$'$ & \MyGradeColor{True}{Positive}{1} & \MyGradeColor{False}{Negative}{1} \\[2.1em]
% & N$'$ & \MyGradeColor{False}{Positive}{1} & \MyGradeColor{True}{Negative}{99} \\
\end{tabular}
}
\subcaption{Evaluation on methods that do not include a JCG}
\end{subfigure}
\hfill
\begin{subfigure}{0.32\textwidth}
\scalebox{0.75}{
\begin{tabular}{c >{\bfseries}r @{\hspace{0.7em}}c 
                @{\hspace{0.4em}}c @{\hspace{0.7em}}l}
  \multirow{10}{*}{\rotatebox{90}{
      \parbox{6.2cm}{\bfseries\centering Actual\\}}} &
      & \multicolumn{2}{c}{\hspace*{-1em}\bfseries Predict} & \\
  & & \bfseries P & \bfseries N \\
  & P$'$ & \MyGradeColor{23}{17} & \MyGradeColor{5}{4} \\[2.1em]
  & N$'$ & \MyGradeColor{41}{30} & \MyGradeColor{67}{49} \\
% & P$'$ & \MyGradeColor{True}{Positive}{1} & \MyGradeColor{False}{Negative}{1} \\[2.1em]
% & N$'$ & \MyGradeColor{False}{Positive}{1} & \MyGradeColor{True}{Negative}{99} \\
\end{tabular}
}
\subcaption{Overall evaluation across the full set of methods}
\end{subfigure}
    \caption[Performance metrics for the vulnerability classification task]%
{Performance metrics for the vulnerability classification task.}
    \label{fig:classification_outcomes}
\end{figure}

\medskip

When the call graph is available, the system correctly identifies 8 vulnerabilities (TP) and misses 2 (FN), while producing 7 false alarms (FP) and 12 correct rejections (TN).  
Without the call graph, the model still detects 5 vulnerabilities and misses 2, but generates a higher number of false positives (9), with 12 true negatives.

Considering the dataset as a whole, the system reports 13 true positives and 24 true negatives, against 16 false positives and 4 false negatives.  
This distribution shows a clear tendency towards recall: the model successfully captures most real vulnerabilities, but at the cost of an increased number of false alarms, particularly when Java-side context is absent.


\begin{table}[ht]
\centering
\begin{tabular}{|p{2cm}|p{2.5cm}|p{2.5cm}|p{2.5cm}|}
\hline 
\centering \textbf{Metric} 
& \centering \textbf{With \glsxtrshort{jcg}} 
& \centering \textbf{Without \glsxtrshort{jcg}} 
& \centering \textbf{All APKs} \tabularnewline
\hline
\hline

\centering \textbf{Accuracy}  
& \centering 70.97~\% 
& \centering 60.71~\% 
& \centering 66.10~\% \tabularnewline

\centering \textbf{Precision}    
& \centering 56.25~\% 
& \centering 35.71~\% 
& \centering 46.67~\% \tabularnewline

\centering \textbf{Recall} 
& \centering 81.82~\% 
& \centering 71.43~\% 
& \centering 77.78~\% \tabularnewline

\centering \textbf{F1-Score}  
& \centering 66.67~\% 
& \centering 47.62~\% 
& \centering 58.33~\% \tabularnewline
\hline
\end{tabular}

\caption{Evaluation metrics computed from the classification results.}
\label{tab:eval_metrics}
\end{table}

Table~\ref{tab:eval_metrics} summarises the overall performance of the system under the two conditions.  
The aggregated accuracy (66.10~\%) indicates that roughly two-thirds of all crash classifications match the ground truth.  

Accuracy is higher when the Java Call Graph is available (70.97~\%) than when it is not (60.71~\%), suggesting that access to Java-to-native context generally improves the reliability of the decisions.  

Precision also benefits from the additional Java-side context.  
In the ``With \glsxtrshort{jcg}'' configuration, 56.25~\% of the crashes flagged as vulnerabilities correspond to true positives, whereas in the ``Without \glsxtrshort{jcg}'' case precision drops to 35.71~\%.  
This indicates that the \gls{jcg} not only increases the fraction of detected vulnerabilities, but also makes positive predictions more trustworthy.  

Recall is consistently high in both settings, but still shows a noticeable gap.  
With \glsxtrshort{jcg}, the system correctly recovers 81.82~\% of all true vulnerabilities, while without it recall drops to 71.43~\%.  
This confirms that providing the \gls{llm} with an explicit \gls{jcg} avoid missing dangerous cases.  

The F1-score, which combines precision and recall into a single metric, highlights the overall impact of the call graph.  
With \glsxtrshort{jcg}, the system reaches 66.67~\%, whereas without it the score falls to 47.62~\%.  
The combined F1-score across all crashes is 58.33~\%, indicating that, on average, the current implementation performs substantially better when \gls{jcg} information is available, and becomes less balanced and less reliable when this context is removed.