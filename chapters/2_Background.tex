\chapter{Background}
\label{chp:backgroud}

\section{Android Native}
Android itself relies extensively on native C/C++ libraries within its core architecture. Many system components and services, such as the runtime, media stack, graphics pipeline, and hardware abstraction layer (HAL), are implemented in native code for reasons of performance, portability, and direct hardware access \cite{android-platform}. The framework exposes parts of this functionality to applications through Java/Kotlin \glspl{api}, enabling managed code to invoke native system services when needed. 

At the system level, Android’s C standard library and dynamic linker are provided by \emph{Bionic}, a lightweight libc/libm/libdl implementation optimised for mobile environments \cite{bionic-maint}. This design allows the platform to combine managed execution (via the \gls{art}) with efficient, low-level components responsible for performance-critical and hardware-dependent operations. Figure~\ref{fig:androidStack} illustrates the Android platform architecture, highlighting how native libraries underpin the runtime, system services, and the HAL.

Beyond the platform itself, developers can extend Android apps with their own native code through the \gls{ndk}.

The advantages of native development stem from the ability to execute compiled machine code directly on the target CPU, allowing finer-grained optimisation and efficient memory use, important for resource-constrained mobile devices. It also enables hardware acceleration and access to specialised APIs (e.g., GPU, DSP, or sensor interfaces), making it indispensable in areas like augmented or virtual reality, game engines (e.g., Unity, Unreal), and device-specific system utilities. \textcolor{red}{Dovrei cercare una citazione}

However, using native code extends the scope of Android applications beyond the safety of the managed runtime. Developers must explicitly handle memory allocation and deallocation, thread synchronisation, and exception propagation, as these are not automatically managed by the \gls{art}. Cross-ABI compatibility, debugging complexity, and maintenance across Android versions further increase the development burden. Consequently, the \gls{ndk} is recommended only when its benefits outweigh these costs \cite{android-ndk-getting-started,android-ndk-concepts,jni-tips}.

This capability of use native code, extends the reach of Android apps beyond the managed runtime but also introduces explicit responsibilities for developers. Native code must handle its own memory management, threading and exception handling. This is significantly different from pure Java/Kotlin development.

\begin{figure}
    \centering
    \begin{tikzpicture}

    \tikzset{
        groupbox/.style={inner sep=0pt}
    }

    \def\W{8cm}
    \def\Wleft{4.45cm}
    \def\Wright{3.45cm}
    \def\Distance{0.8mm}
    
    \node[layer, fill=cSystem!60, minimum width=\W] (apps) {System Apps};
    \node[layer, fill=cJava!60, minimum width=\W, below=\Distance of apps] (java) {Java API Framework};
    
    \node[layer, fill=cNative!60, minimum width=\Wleft, below=\Distance of java, anchor=west] (native) at ($(java.south west)+(0,-3.5mm)$) {Native C/C++ Libraries};
    \node[layer, fill=cRuntime!70, minimum width=\Wright, below=\Distance of java, anchor=east] (runtime) at ($(java.south east)+(0,-3.5mm)$) {Android Runtime};
    
    \node[groupbox, fit=(native) (runtime)] (BOX) {};
    
    \node[layer, fill=cHAL!60, minimum width=\W, below=\Distance of BOX] (hal) {Hardware Abstraction Layer (HAL)};
    \node[layer, fill=cKernel!60, minimum width=\W, below=\Distance of hal] (kernel) {Linux Kernel};
\end{tikzpicture}

    \caption[Android software stack.]%
    {Android software stack \cite{android-platform}}
    \label{fig:androidStack}
\end{figure}

\subsection{\glsxtrlong{ndk}}
The \gls{ndk} is a collection of tools, headers, and libraries that enable developers to embed C and C++ code within Android applications and interact natively with hardware, sensors, and system \glspl{api} \cite{android-ndk-getting-started}. The \gls{ndk} supports compilation into shared (and static) libraries that can be packaged inside the \gls{apk}, and offers native interfaces for tasks such as sensor input, asset loading, and more \cite{android-ndk-concepts}.

Developers typically adopt native code for three main reasons:
\begin{itemize}
  \item \textbf{Performance optimization:} achieving low-latency processing in compute-intensive domains such as graphics, signal processing, physics, or cryptography.
  \item \textbf{Library reuse:} integrating existing C/C++ libraries (e.g.\ cryptography, compression, codecs) to avoid rewriting functionality in Java/Kotlin.
  \item \textbf{Hardware or vendor-specific access:} interacting directly with low-level or proprietary \glspl{api} (e.g.\ custom sensors, specialized accelerators) not exposed in the Java framework.
\end{itemize}

However, using the \gls{ndk} comes with tradeoffs. Native development adds complexity in build configuration, cross-ABI support, debugging, and maintenance across Android versions. Not all Android \glspl{api} are directly available through the \gls{ndk}, so bridging via \gls{jni} is often required for broad framework functionality \cite{android-ndk-concepts}. 

\subsection{\glsxtrlong{jni}}
The \gls{jni} (Java Native Interface) is a native programming interface that enables Java or Kotlin code running in a virtual machine to interoperate with libraries and applications written in C/C++ \cite{jni-spec-design,jni-spec-intro}. It allows managed code to call into native code, and for native code to call back into the \gls{vm}, manipulate Java objects, throw exceptions, and more. The \gls{jni} is designed to impose no restrictions on the implementation of the \gls{vm}, thereby preserving binary compatibility across \gls{vm} vendors \cite{jni-spec-intro}.  

Listing~\ref{lst:javaNative} shows a Java class declaring a \texttt{native} method, while Listing~\ref{lst:JNI} gives the corresponding C implementation on the native side.

\begin{lstlisting}[language=Java+JNI, caption={Java class declaring a \texttt{native} method}, label={lst:javaNative}] 
public class foo {
    private native double bar(int i, String s);
    
    static {
        System.loadLibrary("native-lib");
    }
}
\end{lstlisting}

\begin{lstlisting}[language=C+JNI, caption={Native implementation of \texttt{bar}}, label={lst:JNI}]
jdouble Java_pkg_foo_bar(JNIEnv *env,   // ptr to JNI interface
                         jobject obj,   // "this" pointer
                         jint i,        // first "real" parameter
                         jstring s)     // second "real" parameter
{
    /* Method implementation */
    return 0.0;
}
\end{lstlisting}


\begin{comment}
    \subsection{Vulnerabilities} %  and Impact on the Android Sandbox
Bugs in native memory safety remain the main cause of serious compromise on Android. 
Although the platform’s sandbox assigns each app a distinct \gls{uid} and constrains it further with SELinux policies, a native component compromised inside that boundary still runs with the app’s privileges and can expose any secrets or capabilities already permitted (e.g., tokens, stored content, or keys) \cite{android-app-sandbox,android-selinux}. In practice, this turns local corruption into high-impact data access or control-flow hijacking even without crossing process boundaries.
Memory-safety defects in native libraries, such as buffer overflows, out-of-bounds reads/writes, use-after-free, and double free, belong to well-known categories captured by \gls{cwe}, and they frequently appear in \gls{cve} records \cite{cwe-mitre,cve-overview}. On Android, these errors may manifest as application crashes, or in more severe cases permit arbitrary code execution or data tampering, depending on mitigations and exploitability assumptions \cite{android-native-risk}. 


Attackers often amplify impact by chaining flaws across interfaces. Many system services are native and reachable over Binder; memory corruption in such a service can be triggered via \gls{ipc} by a malicious client to escape app-level limits \cite{mao-nass-usenix25}. 

In other scenarios, native code may corrupt memory pages shared with the managed runtime (e.g. altering \gls{jni} or class metadata) by transgressing memory protection via system calls (e.g. \texttt{mprotect}), thus undermining Java-level safety invariants and enabling code injection or VM subversion \cite{going-native-ndss17}. 
Another common vector is outdated third-party native libraries: empirical studies show that a significant fraction of popular apps ship unpatched vulnerable native libraries, and that developers take on average hundreds of days to apply upstream patches \cite{almanee-icse21}. This “patch latency” enables attackers to exploit known vulnerabilities even in apps that seem benign at the Java layer.  

Mitigation requires defense-in-depth: modern Android hardening (ASLR/DEP, control-flow integrity, memory tagging), combined with strict interface design and timely library updates, can substantially raise exploitation cost even when bugs persist \cite{android-ndk-mte}. 
\end{comment}

\subsection{Vulnerabilities and Impact on the Android Sandbox}
Native memory-safety bugs remain a dominant cause of serious compromise on Android. Although the platform’s sandbox assigns each app a distinct \gls{uid} and further constrains it with SELinux policies, a native component compromised inside that boundary still runs with the app’s privileges and can expose secrets or capabilities already permitted (e.g., tokens, stored content, keys) \cite{android-app-sandbox,android-selinux}. In practice, this turns local corruption into high-impact data access or control-flow hijacking even without crossing process boundaries.

Memory-safety defects in native libraries, such as buffer overflows, out-of-bounds reads/writes, use-after-free, and double free, belong to well-known \gls{cwe} families and frequently appear in \gls{cve} records \cite{cwe-mitre,cve-overview}. On Android, these errors may cause crashes or, in more severe cases, permit arbitrary code execution or data tampering depending on mitigations and exploitability assumptions \cite{android-native-risk}.

Attackers often amplify impact by chaining flaws across interfaces. Many system services are native and reachable over Binder; memory corruption in such a service can be triggered via \gls{ipc} by a malicious client to bypass app-level limits \cite{mao-nass-usenix25}. In other scenarios, native code may corrupt memory shared with the managed runtime (e.g., altering \gls{jni} or class metadata) by abusing memory-protection system calls, undermining Java-level safety invariants and enabling code injection or \gls{vm} subversion \cite{going-native-ndss17}. Another common vector is outdated third-party native libraries: empirical studies show that a significant fraction of popular apps ship unpatched vulnerable native libraries, and that developers take on average hundreds of days to apply upstream patches \cite{almanee-icse21}. This patch latency enables exploitation of known bugs even in apps that appear benign at the Java layer.

Mitigation requires defense-in-depth: modern Android hardening (\gls{aslr}/\gls{dep}, control-flow integrity, memory tagging), combined with strict interface design and timely library updates, can significantly reduce the feasibility and reliability of exploitation even when bugs persist \cite{android-ndk-mte}.

\section{Fuzzing}
Fuzzing is an automated testing technique that repeatedly executes a target program with many inputs, either randomly generated or systematically derived, to expose defects such as crashes, assertion failures, or memory-safety violations. Modern fuzzers prioritise inputs that increase code coverage (e.g., edge or block coverage), and couple execution with hardening oracles (e.g., sanitizers) to turn latent memory errors into reliable, actionable signals \cite{sutton2007fuzzing,miller1990fuzz}.

\subsection{Types}
A fuzzer can be classified along three ways \cite{sutton2007fuzzing,godefroid2008sage}:
\begin{itemize}
  \item \textbf{Input production:} \emph{generation-based} where inputs are produced from a model or specification and \emph{mutation-based} where inputs are produced by mutating existing seeds. Coverage-guided mutation fuzzers (e.g., \gls{aflpp}, libfuzzer) iteratively mutate seeds that increase coverage.
  \item \textbf{Input knowledge:} \emph{structured}, grammar- or model-based, aware of input format/protocol, or \emph{unstructured}, no knowledge of structure; “dumb” or purely random.
  \item \textbf{Program knowledge:} \emph{black-box}: no visibility into internals, \emph{grey-box}: lightweight instrumentation, e.g., edge coverage, and \emph{white-box}: constraint solving/symbolic execution to systematically steer execution, as in SAGE \cite{godefroid2008sage}.
\end{itemize}

\subsection{Uses}
Fuzzing is primarily employed to discover security-relevant defects in safety- or security-critical software, especially memory-safety bugs in native code. Its strength lies in \emph{demonstrating the presence} of bugs via concrete, reproducible inputs. Proving correctness for all inputs requires a formal specification and formal methods; fuzzing complements, rather than replaces, such approaches \cite{sutton2007fuzzing}. In practice, fuzzing integrates into secure development lifecycles and CI pipelines to continuously test libraries, parsers, and interfaces that process untrusted data.

\subsection{Android context (sanitizers as oracles).}
On Android, fuzzing native components benefits from memory error detectors such as \gls{hwasan} and, historically, \gls{asan}. These sanitizers instrument code to detect out-of-bounds accesses, use-after-free, and related violations during fuzzing, producing precise diagnostics that accelerate crash triage \cite{hwasan-ndk,asan-aosp,asan-clang}.

Two widely used coverage-guided engines are \gls{aflpp} (a fork and extension of AFL with a rich ecosystem of mutators and instrumentation options) and \gls{libfuzzer}. Both have been successfully applied to native libraries, including on Android, to elicit high-quality crashes and to maximise coverage under realistic budgets \cite{aflpp-woot20,libfuzzer-llvm}.






\section{\glsxtrlong{llm}}
\glspl{llm} are deep neural networks based on the Transformer architecture, which replaces recurrence/convolution with stacked self-attention and feed-forward layers to model long-range dependencies efficiently \cite{vaswani2017attention}. In practice, today’s \glspl{llm} are pre-trained on large corpora via next-token prediction and then adapted to follow user prompts. 

\subsection{Limitations and Hallucinations}
Because \glspl{llm} predict tokens rather than verify facts, they can generate confident but incorrect content (``hallucinations''). Surveys and recent studies document multiple forms (e.g., intrinsic or confabulated outputs) and show that even state-of-the-art models may produce fluent fabrications \cite{ji2023hallucination,farquhar2024semanticentropy}. In security-oriented workflows such as vulnerability triage, this can manifest as invented \gls{cve} identifiers, non-existent \glspl{api} or packages, fabricated proof-of-concept details, or misattributed stack traces, errors that can mislead severity assessment or waste analyst time. Empirical work on code-generation further highlights ``package hallucinations'', where models recommend non-existent libraries, creating supply-chain risk if an attacker later publishes a malicious package with that name \cite{spracklen2025package}. Consequently, any use of \glspl{llm} for triage should treat outputs as hypotheses to be checked against authoritative sources and artefacts (code, binaries, advisories), and prefer designs that promote verifiability and uncertainty signalling rather than unconditional trust.





\section{\glsxtrlong{mcp}}
\gls{mcp} is an open protocol, introduced by Anthropic (creator of Claude) in November 2024, that standardises how AI applications obtain and manage external context via a client–server architecture.
An \emph{MCP host} (the AI application) connects to one or more \emph{MCP servers} through dedicated \emph{MCP clients}, enabling the host to discover and use tools, read resources, and apply prompts exposed by servers. The protocol separates a JSON-RPC–based \emph{data layer} (primitives, lifecycle, notifications) from a \emph{transport layer} (e.g., stdio for local, streamable HTTP for remote), so the same message semantics work across local and remote deployments \cite{mcp-architecture,mcp-intro}. In brief, servers contribute context and actions; clients manage capability negotiation and exchanges; the host orchestrates everything in the user-facing application \cite{mcp-architecture,mcp-server-concepts,mcp-client-concepts}. Figure~\ref{fig:mcp-simple-diagram} provides a high-level view of this host–client–server pattern and example integrations.

\begin{figure}
    \centering
    \scalebox{0.8}{\begin{tikzpicture}[font=\small, node distance=5mm and 16mm]

% --- Stili per i box principali ---
\tikzset{
  groupbox/.style={inner sep=0pt},
  grouplabel/.style={font=\footnotesize\bfseries, fill=white, inner sep=1pt}
}

\def\WNode{45mm}
\def\HNode{25mm}
\def\NDistanceH{10mm}
\def\NDistanceV{2mm}

% =============== BLOCCO SINISTRA (SX): 3 elementi ===============
\node[layer, minimum width=\WNode] (L1) {\textbf{Chat interface}\\Claude Desktop};
\node[layer, minimum width=\WNode, below=\NDistanceV of L1] (L2) {\textbf{IDEs and code editors}\\Claude Code, Goose};
\node[layer, minimum width=\WNode, below=\NDistanceV of L2] (L3) {\textbf{Other AI applications}\\5ire, Superinterface};
% Box che "abbraccia" i tre nodi: il suo centro è a metà verticale
\node[groupbox, fit=(L1) (L3), label={[grouplabel, label distance=\NDistanceV]south:\gls{mcp} Hosts}] (SXBOX) {};

% =============== BLOCCO CENTRALE (CENTRO): 1 elemento ===========
% Posizionato "right=..." RISPETTO A SXBOX (non L1): così condivide la stessa y del centro
\node[layer, fill=cHAL!60, minimum width=\WNode, minimum height=\HNode, right=\NDistanceH of SXBOX] (C1)
  {\textbf{MCP}\\Standardized protocol};
\node[groupbox, fit=(C1)] (CTBOX) {};

% =============== BLOCCO DESTRA (DX): 3 elementi =================
% Posiziona il blocco destro a destra del nodo centrale (che è ora centrato verticalmente)
\node[layer, minimum width=\WNode, right={\NDistanceH*2+\WNode} of L1] (R1) {\textbf{Data and file systems}\\PostgreSQL, SQLite, GDrive};
\node[layer, minimum width=\WNode, below=\NDistanceV of R1] (R2) {\textbf{Development tools}\\Git, Sentry, etc.};
\node[layer, minimum width=\WNode, below=\NDistanceV of R2] (R3) {\textbf{Productivity tools}\\Slack, Google Maps, etc.};
\node[groupbox, fit=(R1) (R3), label={[grouplabel, label distance=\NDistanceV]south:\gls{mcp} Server}] (DXBOX) {};

% --- Stile frecce ---
\tikzset{
  bidi/.style={
    draw=black!35,
    line width=0.5pt,
    rounded corners=2mm,
    {Latex[length=2mm]}-{Latex[length=2mm]}
  },
}


\def\ADistanceH{\NDistanceH*0.5}
\def\ADistanceV{6mm}

\draw[bidi] (L1.east) -|  ([shift=({-\ADistanceH, \ADistanceV})]C1.west) -- ([yshift={\ADistanceV}]C1.west);
\draw[bidi] (L2.east) -- (C1.west);
\draw[bidi] (L3.east) -|  ([shift=({-\ADistanceH, -\ADistanceV})]C1.west) -- ([yshift={-\ADistanceV}]C1.west);


\draw[bidi] ([yshift={\ADistanceV}]C1.east) --  ([shift=({\ADistanceH, \ADistanceV})]C1.east) |- (R1.west);
\draw[bidi] (C1.east) -- (R2.west);
\draw[bidi] ([yshift={-\ADistanceV}]C1.east) --  ([shift=({\ADistanceH, -\ADistanceV})]C1.east) |- (R3.west);




\end{tikzpicture}
}
    \caption[High-level MCP architecture.]%
      {High-level \gls{mcp} architecture. The \gls{mcp} host (e.g., chat interface or IDE) coordinates one or more \gls{mcp} clients, each connected to an \gls{mcp} server that exposes tools, resources, and prompts \cite{mcp-intro}.}
    \label{fig:mcp-simple-diagram}
\end{figure}



\begin{comment}
    \subsection{Core Concepts}
\paragraph{Data layer and primitives.}
The data layer defines message schemas and semantics (JSON-RPC 2.0) and the core \emph{primitives} that servers expose to clients: \emph{tools} (executable functions that the AI can call), \emph{resources} (read-only context like files, DB records, API responses), and \emph{prompts} (reusable instruction templates). Clients discover and retrieve these via `*/list` and `*/get`, and invoke tools with `tools/call`. Lifecycle management handles initialization and capability negotiation; notifications propagate dynamic changes such as tool-list updates \cite{mcp-architecture,mcp-server-concepts}.

\paragraph{Transport layer.}
Transports carry the same data-layer messages. \emph{Stdio} provides low-latency local connections; \emph{streamable HTTP} enables remote servers with standard auth (API keys, bearer tokens, OAuth). Transport concerns (framing, auth) are kept orthogonal to the data schema \cite{mcp-architecture}.
\end{comment}

\subsection{MCP Host}
MCP Host is the AI application that coordinates and manages one or multiple \gls{mcp} clients. It maintains user experience, orchestrates connections to configured servers, and routes model/tool interactions \cite{mcp-architecture}.  

On startup, the host instantiates one client per target server, runs initialization (protocol version and capabilities), caches discovered primitives, and enforces the app’s permission and UX policies when tools or elicitation are used \cite{mcp-architecture,mcp-client-concepts}.

\subsection{MCP Client}
MCP Client is a component that maintains a one-to-one connection to an \gls{mcp} server and obtains context for the host to use \cite{mcp-architecture}.  

The client executes the data-layer protocol: initialization, discovery (`*/list`), retrieval (`*/get`), tool execution (`tools/call`), and handling notifications. Clients may also offer \emph{client primitives} to servers, \emph{sampling} (request an \gls{llm} completion via the host), \emph{elicitation} (ask the user for inputs/confirmation), and \emph{logging}, which enable richer, permissioned workflows without coupling servers to a specific model SDK \cite{mcp-client-concepts,mcp-architecture}.

\subsection{MCP Server}
MCP Server is a program that provides context to \glspl{mcp} client via standardised interfaces \cite{mcp-architecture}.  

Servers expose \emph{tools}, \emph{resources}, and \emph{prompts} described with schemas (e.g., JSON Schema for tool inputs/outputs). Typical servers include filesystem, databases, code hosts, observability or messaging platforms. Server capabilities are declared during initialization, can change at runtime (with notifications), and are discoverable by clients \cite{mcp-server-concepts,mcp-architecture}.

\subsection{Elicitation}
Elicitation in \gls{mcp} allows a server to request the exact user inputs still required \emph{during} an ongoing operation. The server pauses a tool call, asks the \gls{mcp} client—subject to the host’s UX and permission policies—to gather typed, structured fields such as a missing identifier, a disambiguation choice, or consent, then resumes processing once the validated payload is returned. Requests are typically described with JSON Schema, are transport-agnostic, and support multi-turn workflows without coupling to any specific UI or model SDK. The host controls presentation and policy enforcement, and servers are expected to avoid sensitive data. In practice, elicitation reduces failures from incomplete inputs and turns uncertainty into explicit, auditable user decisions. \cite{mcp-architecture,mcp-elicitation}

\begin{figure}
    \centering
    \includegraphics[width=0.8\linewidth, trim={-3.5cm 0 3.5cm 0},clip]{figures/elicitation-flow.png}

    \caption[Elicitation flow of MCP]%
    {Elicitation flow of \gls{mcp} \cite{mcp-client-concepts}}
    \label{fig:elicitation-flow}
\end{figure}

Figure~\ref{fig:elicitation-flow} shows the server-initiated sequence: the server issues \texttt{elicitation/create}, the host presents an interface, the user responds, the client completes the request, and the server continues with the new context. \cite{mcp-elicitation}


\section{Vulnerability Scoring Systems}
\subsection{\glsxtrlong{cvss}}
\subsection{\glsxtrlong{cwe}}
\subsection{\glsxtrlong{cve}}


\section{Reverse engineering}
\subsection{Ghidra}
\subsection{Jadx}


\begin{comment}


\section{Memory safety differences and vulnerability classes}
Java/Kotlin provide memory safety by construction (bounds checks, GC), whereas C/C++ delegate memory and lifetime management to the programmer, enabling classes of bugs that frequently underlie security vulnerabilities. Common weakness families include buffer overflow/underflow, use-after-free, out-of-bounds read/write, and double free; these are captured in \gls{cwe} and repeatedly observed in \gls{cve} records \cite{cwe-mitre,cve-overview}. On Android, such defects can lead to denial of service (crash) or arbitrary code execution, depending on mitigations and exploitability \cite{android-native-risk}.



\subsection{Fuzzing native libraries and harness generation}
Coverage-guided fuzzing mutates inputs to maximise code coverage while detecting crashes via sanitizers or OS signals. \gls{aflpp} generalises AFL’s greybox approach and incorporates state-of-the-art features (e.g., custom mutators) for broad applicability \cite{aflpp-woot20}. \gls{libfuzzer} integrates tightly with LLVM sanitizers and is widely used for in-process targets \cite{libfuzzer-llvm}. Sanitizers such as \gls{asan} and \gls{hwasan} are available on Android and provide precise diagnostics for memory-safety violations \cite{asan-android-aosp,hwasan-ndk}.

The limiting factor for library fuzzing is the harness: an executable environment that exercises target \glspl{api} with realistic sequences and well-formed objects. Automating harness synthesis for Android native libraries is challenging due to cross-language interactions, complex object lifecycles, and the need to emulate app-specific call sequences via \gls{jni}. POIROT addresses these challenges by statically analysing an app’s Java-side usage of the native library, synthesising a consumer-specific harness that supports bidirectional \gls{jni}, and running large-scale fuzzing campaigns \cite{poirot-usenix25}. Prior works on harness generation and Android library fuzzing (e.g., ATLAS, FuzzGen) highlight the broader landscape and the difficulty of scaling without app-specific context \cite{atlas-issta24,fuzzgen-usenix20}.

\subsection{Crash triage: principles and challenges}
Crash triage seeks to (i) deduplicate and bucket crashes, (ii) decide whether a crash indicates a security-relevant vulnerability, and (iii) prioritise fixes. Heuristic exploitability assessors and semantic crash bucketing methods have been proposed to improve accuracy and scalability, yet they typically assume desktop/server contexts and still require expert review \cite{scb-ase18,igor-ccs21}. On Android, crashes often originate in native code called through \gls{jni} and may depend on app-specific object invariants; reproducing and interpreting them benefits from reverse-engineering and sanitizer evidence \cite{poirot-usenix25,asan-android-aosp}.

\section{\gls{llm}s with tool grounding for security triage}
Emerging studies suggest that \gls{llm}s can assist in vulnerability management tasks such as \gls{cwe} classification and severity estimation. For instance, CASEY reports non-trivial accuracy for \gls{cwe} and severity identification on proprietary datasets, indicating potential to streamline triage workflows \cite{casey-aic25}. LProtector explores \gls{llm}-driven detection pipelines that combine static/dynamic signals with model judgements \cite{lprotector-2024}. However, unguided \gls{llm}s may hallucinate or misinterpret low-level details. Tool grounding mitigates this risk by connecting the model to program artefacts through \gls{mcp}, enabling retrieval of Jadx bytecode contexts, Android manifests, and Ghidra decompilation/disassembly snippets that can be quoted as evidence in triage outputs \cite{mcp-overview,jadx-github,ghidra-github}.

\section{Taxonomies and severity scoring}
To communicate results and integrate with established workflows, crash judgements should map to standard taxonomies. \gls{cwe} provides a community-developed catalogue of weakness types \cite{cwe-mitre}; the \gls{cve} programme assigns identifiers to specific vulnerabilities \cite{cve-overview}. Severity can be expressed using \gls{cvss} v3.1 or v4.0, which define reproducible scoring formulas and vector strings suitable for automation \cite{cvss31,cvss40}. Our pipeline adopts these standards so that triage outputs are actionable for security engineering teams and compatible with existing disclosure and patch processes.

\section*{Provisional references for the Background}
Android app model and \gls{jni} \cite{art-doc,ndk-guide,jni-book,jni-tips}; memory-safety risks \cite{android-native-risk,cwe-mitre,cve-overview}; fuzzing and sanitizers \cite{aflpp-woot20,libfuzzer-llvm,asan-android-aosp,hwasan-ndk}; POIROT and harness generation landscape \cite{poirot-usenix25,atlas-issta24,fuzzgen-usenix20}; crash bucketing/triage \cite{scb-ase18,igor-ccs21}; tool grounding and supporting tools \cite{mcp-overview,jadx-github,ghidra-github}; severity/taxonomies \cite{cvss31,cvss40}.

    
\end{comment}