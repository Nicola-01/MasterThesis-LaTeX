\chapter{Related Work}
\label{chp:relatedWork}

\begin{comment}
    
\section{Fuzzing of Android native libraries and harness generation}
Fuzzing of Android native libraries poses challenges beyond conventional user-space fuzzing due to the \gls{jni} boundary, lifecycle constraints, and the need for realistic cross-language call sequences. POIROT automatically synthesises consumer-specific harnesses for closed-source Android libraries by analysing Java-side usage and supporting bidirectional \gls{jni} interactions, enabling large-scale campaigns that uncovered thousands of unique crashes \cite{poirot-usenix25}. Atlas similarly targets Android closed-source native libraries with a cross-language fuzzing framework and automatic harness generation \cite{atlas-issta24}. Beyond the Android setting, FuzzGen synthesises library-specific fuzzers by inferring API contracts and integrating with \gls{libfuzzer} to reach deep states \cite{fuzzgen-usenix20}. Prior engineering work shows that reproducing \gls{art} behaviour and \gls{jni} semantics is critical for validity and reproducibility of results \cite{polito-android-native-fuzzing}. As to fuzzing engines, \gls{aflpp} extends greybox fuzzing with a rich ecosystem of mutators and instrumentation \cite{aflpp-woot20}, while \gls{libfuzzer} offers tight integration with LLVM sanitizers for in-process fuzzing \cite{libfuzzer-llvm}. Sanitizers such as \gls{asan} and \gls{hwasan} are widely used to diagnose memory-safety defects during fuzzing on Android \cite{asan-android-aosp,hwasan-ndk}.

\section{Crash triage and LLM-based vulnerability classification}
Traditional crash triage combines bucketing, deduplication, and heuristic \emph{exploitability} estimation, but typically requires expert oversight and offers limited precision across diverse crash types \cite{scb-ase18,igor-ccs21}. Recent studies investigate \gls{llm}-assisted triage. CASEY reports non-trivial accuracy for CWE classification (68\%) and severity identification (73.6\%) on an NVD-derived corpus, indicating that \gls{llm}s can streamline parts of vulnerability triage \cite{casey-aic25}. LProtector explores \gls{llm}-driven vulnerability detection for C/C++ projects with retrieval augmentation, highlighting benefits and pitfalls of model-in-the-loop security analysis \cite{lprotector-2024}. Broader surveys benchmark \gls{llm}s and agents for practical software security, consolidating evidence that model judgements improve when provided with structured context and repository-level signals, but also documenting variability across tasks and datasets \cite{acl25-llm-benchmark-slr,slr-llm-vuln-2025}. These findings motivate tool-grounded designs for triage under Android/\gls{jni} constraints.

\section{Tool grounding with \gls{mcp} for program analysis and reverse engineering}
Tool grounding aims to reduce hallucinations and improve faithfulness by letting the \gls{llm} query authoritative artefacts. The \gls{mcp} standardises how models connect to external tools and data sources \cite{mcp-overview,anthropic-mcp}. In our setting, grounding uses Jadx to retrieve bytecode/manifest context and Ghidra for disassembly/decompilation. An emerging ecosystem of \gls{mcp} servers exposes reverse-engineering capabilities of Ghidra to \gls{llm}-based clients (e.g., symbol and function listings, decompilation), facilitating evidence-linked reasoning \cite{ghidra-mcp-laurie,ghidra-mcp-suid}. While promising, the security surface of tool integrations must be considered; recent supply-chain incidents around \gls{mcp} servers highlight the need for permission scoping and integrity checks \cite{mcp-supplychain-incident}. Our design leverages \gls{mcp} to fetch verifiable snippets (stack frames, function names, minimal decompiled code) that the triager references in its rationale.

\section{Positioning}
Compared to POIROT and Atlas, which focus on automating harness generation and fuzzing at scale \cite{poirot-usenix25,atlas-issta24}, our work targets the \emph{post-fuzzing} triage stage for Android native crashes. Relative to CASEY/LProtector and survey results on \gls{llm}-for-security \cite{casey-aic25,lprotector-2024,acl25-llm-benchmark-slr,slr-llm-vuln-2025}, our contribution is a tool-grounded pipeline that couples \gls{mcp}-mediated access to Jadx/Ghidra with crash artefacts from POIROT, aiming to improve precision, explainability, and analyst verification for \gls{jni}-mediated memory-safety issues.

\end{comment}